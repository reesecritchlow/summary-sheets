\documentclass{article}
\usepackage[utf8]{inputenc}
\usepackage[letterpaper, portrait, margin=0.75in]{geometry}
\usepackage{multicol}



\begin{document}

\begin{center}
    \begin{Large}
        Series and Series Tests
    \end{Large}\\
    Reese Critchlow, 2021
\end{center}

\begin{large}
    Geometric Series
\end{large}\\
\noindent\rule{\textwidth}{0.5pt}\smallskip\\
\indent \underline{Infinite Sum}
\begin{multicols}{4}
\begin{center}
    General Form
    \[
        S = \sum_{n=1}^{\infty}ar^{n-1}
    \]
    
    Value if $|r| < 1$
    \[
        S = \frac{a}{1-r}
    \]
    
    Value if $r = 1$
    \[
        S = \textrm{Divergent}
    \]
    
    Value if $|r| > 1$
    \[
        S = \textrm{Divergent}
    \]
\end{center}
\end{multicols}

\indent \underline{Partial Sum}
\begin{multicols}{3}
\begin{center}
    General Form
    \[
        S_N = \sum_{n=1}^{N}ar^{n-1}
    \]
    
    Value if $|r| \neq 1$
    \[
        S_N = a\frac{1-r^{N+1}}{1-r}
    \]
    
    Value if $r = 1$
    \[
        S_N = a(N+1)
    \]
    
\end{center}
\end{multicols}


\begin{large}
    Telescoping Series
\end{large}\\
\noindent\rule{\textwidth}{0.5pt}
\begin{multicols}{2}
    \underline{Infinite Sum}\\\columnbreak
    \underline{Partial Sum}
\end{multicols}

\begin{multicols}{4}
\begin{center}
    General Form 
    \[
        S = \sum_{n = 1}^{\infty}{a_n - a_{n+1}}
    \]
    
    Value
    \[
        S = a_1 - \lim_{n \to \infty}{a_n}
    \]
    
    General Form 
    \[
        S_N = \sum_{n = 1}^{N}{a_n - a_{n+1}}
    \]
    
    Value
    \[
        S = a_1 - {a_{N+1}}
    \]
    

    
\end{center}
\end{multicols}

\begin{large}
    Divergence Test
\end{large} -- Best for when the n\textsuperscript{th} term in the series \textit{fails} to converge to zero towards infinity\\
\noindent\rule{\textwidth}{0.5pt}
\begin{multicols}{2}
\begin{center}
    \underline{Theorem}
    \medskip
    \[
        \textrm{\textit{if  }} \lim_{n \to \infty}a_n \neq 0
    \]
    The series diverges.
    \[
        \textrm{\textit{if  }} \lim_{n \to \infty}a_n = 0
    \]
    The test is completely fucking useless.
    \vfill\null\columnbreak
    \underline{Conditions}\\\bigskip
    No Conditions
    \end{center}
\end{multicols}

\begin{large}
    The Integral Test
\end{large} -- Best for when $n$ can be easily substituted for $x$ and integrated\\
\noindent\rule{\textwidth}{0.5pt}
\begin{multicols}{2}
\begin{center}
    \underline{Theorem}
    \medskip
    \[
         \texterm{\textit{if }} \int_{N_0}^\infty{f(x)dx} \textrm{  converges, then }  \sum_{n = 1}^\infty{a_n}\textrm{  converges}
    \]
     - and -
    \[
        \texterm{\textit{if }} \int_{N_0}^\infty{f(x)dx} \textrm{  diverges, then }  \sum_{n = 1}^\infty{a_n}\textrm{  diverges}
    \]

    \vfill\null\columnbreak
    \underline{Conditions}
    \end{center}
    \begin{enumerate}
        \item $f(x) \geq 0$ for all $x \geq N_0$
        \item $f(x)$ is a decreasing function
        \item $f(n) = a_n$ for all $n \geq N_0$
    \end{enumerate}
    \begin{center}
    \[
        \int_N^\infty f(x)dx \leq \sum_{n=N}^\infty \leq a_N + \int_N^\infty f(x)dx \textrm{  or} \int_{N-1}^\infty f(x)dx
    \]
    \end{center}
\end{multicols}
\pagebreak
\begin{large}
    The P Test
\end{large}\\
\noindent\rule{\textwidth}{0.5pt}

\begin{multicols}{2}
\begin{center}
    \underline{Theorem}
    \medskip
    \[
         \textrm{Given } S = \sum_{n = 1}^\infty{\frac{1}{n^p}}
    \]
    if $p > 1$, then $S$ converges.\\\smallskip
    if not, $S$ is divergent.
    \\
    \vfill\null\columnbreak
    \underline{Conditions}
    \\\bigskip
    No Conditions
    \end{center}
\end{multicols}

\begin{large}
    The Comparison Test
\end{large} -- Best when $a_n$ can be easily simplified to a term, $b_n$ at $n$ very large\\
\noindent\rule{\textwidth}{0.5pt}
\begin{center}
\underline{Theorem}

\begin{multicols}{2}\noindent
    \[
         \textrm{Given } S = \sum_{n = 1}^\infty{a_n}
    \]
    
    \[
         \textrm{and some } S_c = \sum_{n = 1}^\infty{c_n}
    \]
    \\
    \vfill\null\columnbreak
    
    \begin{enumerate}
        \item if $|a_n| < c_n$ for all $n > N$, and $\sum_{n=1}^\infty{c_n}$\\ converges, then $\sum_{n = 1}^\infty{a_n}$ converges
        \item if $a_n > c_n$ for all $n > N$, and $\sum_{n=1}^\infty{c_n}$\\ diverges, then $\sum_{n = 1}^\infty{a_n}$ diverges
    \end{enumerate}
    
    \noindent Note: it is always a good choice to compare with a p-test-able sum.
    
\end{multicols}
\end{center}

\begin{large}
    The Limit Comparison Test
\end{large} -- Same applications as the \textit{Comparison Test}, but with a more definitive result\\\indent\hspace{2.13in} and more general use.\\
\noindent\rule{\textwidth}{0.5pt}
\begin{center}
    \underline{Theorem}

\begin{multicols}{2}\noindent
    Given
    \begin{multicols}{2}\noindent
    \[
         S_a = \sum_{n = 1}^\infty{a_n}
    \]
    \[
         S_b = \sum_{n = 1}^\infty{b_n}
    \]
    \end{multicols}
    and the limit
    \medskip
    \[
        L = \lim_{n \to \infty}{\frac{a_n}{b_n}} \neq \textrm{DNE}
    \]
    
    \vfill\null\columnbreak

    \begin{enumerate}
        \item if $S_b$ converges, then $S_a$ converges
        \item if \underline{$L \neq 0$} and $S_b$ diverges, then $S_a$ diverges.\\\medskip
        \indent Note: The Condition $L \neq 0$ is \underline{crucial}! 
        \item When $L = 0$, no information can be provided on divergence.
    \end{enumerate}

\end{multicols}
\end{center}

\begin{large}
    Alternating Series Test
\end{large}\\
\noindent\rule{\textwidth}{0.5pt}

\begin{center}
\underline{Theorem}

\begin{multicols}{2}\noindent
    
    The sum
    \[
        \sum_{n = 1}^\infty{(-1)^{n-1}b_n}
    \]
    
    
    \vfill\null\columnbreak

    Converges \underline{if}
    \begin{enumerate}
        \item $b_n \geq b_{n+1}$ for $n \geq N$
        \item $\lim_{n \to \infty}{b_n} = 0$
    \end{enumerate}
    \end{multicols}
\end{center}

\begin{large}
    Remainders
\end{large}\\
\noindent\rule{\textwidth}{0.5pt}

\begin{center}
\begin{multicols}{2}\noindent
    \underline{Partial Sum Error (Alternating)}
    \[
        |R_n| \leq a_{N+1}
    \]
    \vfill\null\columnbreak
    \underline{Partial Sum Error -- Integral Test}\\
    \[
    \int_{N+1}^\infty f(x)dx \leq S_\infty - S_N \leq \int_N^\infty{f(x) dx}
    \]
    
\end{multicols}
\end{center}

\begin{large}
    Conditional and Absolute Convergence
\end{large}\\
\noindent\rule{\textwidth}{0.5pt}
\begin{multicols}{2}
\begin{center}
    \underline{Absolute Convergence}\\
    \medskip
    
    Given a series $\sum_{n = 1}^\infty a_n$, if $\sum_{n = 1}^\infty |a_n|$ converges (the absolute value of that series), then $\sum_{n = 1}^\infty a_n$ converges.\\
    
    \vfill\null\columnbreak
    
    \underline{Conditional Convergence}\\\medskip
    
    If a series $\sum_{n=1}^\infty a_n$ converges, but its absolute value, $\sum_{n = 1}^\infty |a_n|$ diverges, then the series is conditionally convergent.
    
\end{center}
\end{multicols}

\begin{large}
    The Ratio Test
\end{large}\\
\noindent\rule{\textwidth}{0.5pt}

\begin{multicols}{2}
\begin{center}
    \underline{Given}
    \[
        \lim_{n \to \infty}{\frac{|a_{n+1}|}{|a_n|}} = L
    \]
    \footnotesize Where $a_n \neq 0$\normalsize
    
    \vfill\null\columnbreak
    \underline{Conclusions}
     
    \begin{enumerate}
        \item If $L < 1$, then the sum is converges (absolutely).
        \item If $L > 1$ or DNE, the sum diverges.
        \item IF $L = 1$, use a different test.
    \end{enumerate}
\end{center}
\end{multicols}

\begin{large}
    Power Series
\end{large}\\
\noindent\rule{\textwidth}{0.5pt}
\begin{multicols}{3}
\noindent
\begin{center}
Exponential
\[
    e^x = \sum_{n = 0}^\infty\frac{x^n}{n!}
\]
Geometric
\[
    \frac{1}{1-x}=\sum_{n = 0}^\infty x^n
\]
\vfill\null\columnbreak
Geometric -- Derivative
\[
    \frac{1}{(1-x)^2}=\sum_{n = 1}^\infty nx^{n-1}
\]
Geometric -- Integral
\[
    \ln(1-x) = -\sum_{n = 0}^\infty\frac{x^{n+1}}{n+1}
\]
Geometric -- Negative Integral
\[
    \ln(1+x) = \sum_{n = 0}^\infty(-1)^n\frac{x^{n+1}}{n+1}
\]
\vfill\null\columnbreak
Sine
\[
    \sin x = \sum_{n = 0}^\infty(-1)^n\frac{x^{(2n+1)}}{(2n+1)!}
\]
Cosine
\[
    \cos x = \sum_{n = 0}^\infty (-1)^n\frac{x^{2n}}{(2n)!}
\]
Arctangent
\[
    \arctan x = \sum_{n = 0}^\infty (-1)^n \frac{x^{2n+1}}{2n+1}
\]
\end{center}
\end{multicols}

\begin{large}
    Taylor Series
\end{large}\\
\noindent\rule{\textwidth}{0.5pt}
\begin{multicols}{3}
\noindent

\begin{center}
    \underline{General Formula}
    \[
        T_N(x) = \sum_{n = 1}^N\frac{f^{(n)}(c)}{n!}(x-c)^n
    \]
    
    \vfill\null\columnbreak
    
    \underline{Error Formula}
    \[
        R_N(x) \leq \frac{f^{(N+1)}(\bar{c})}{(N+1)!}(x-c)^{N+1}
    \]
    
    \vfill\null\columnbreak
    
    \underline{Remark}\\
        \item If $R_N(x)$ reaches zero for some $N$, the series is said to be \underline{convergent}. This can also be proven by using the ratio test on the series. When a Taylor Series converges, the approximation equals the function when $N \to \infty$.
\end{center}

\end{multicols}

\begin{large}
    Binomial Expansions
\end{large}\\
\noindent\rule{\textwidth}{0.5pt}

\begin{multicols}{2}
\begin{center}
\underline{General Form}

\[
    (a+b)^n = \sum_{k = 0}^n {n \choose k} a^{n-k}b^k
\]

Where:
\[
    {n \choose k} = \frac{n!}{k!(n-k)!}
\]
\end{center}
\end{multicols}


\end{document}
