\documentclass{article}
\usepackage[utf8]{inputenc}
\usepackage[letterpaper, portrait, margin=1in]{geometry}
\usepackage{multicol}
\usepackage{amsmath}
\usepackage{amssymb}
\usepackage{enumerate}
\setlength\parindent{0pt}
\usepackage{enumerate}
\usepackage{graphicx}
\graphicspath{ {./images/} }
\usepackage{fancyhdr}
\usepackage{tcolorbox}
\hyphenchar\font=-1
\usepackage{tabularx}

\newcommand{\header}[1]{\begin{large}\noindent #1\end{large}\\\rule{\textwidth}{0.5pt}}
\newcommand{\sheader}[1]{\underline{#1:}}
\newcommand{\mheader}[1]{\textbf{{\sheader{#1}}}}
\newcommand{\gap}{\medskip\\}
\newcommand{\centertext}[1]{\begin{center}#1\end{center}}
\newcommand{\bfrac}[2]{\left(\frac{#1}{#2}\right)}
\newcommand{\formula}[3]{\begin{center} \begin{tcolorbox}[title = #2] $$#3$$\end{tcolorbox}\end{center}}
\newcommand{\where}{\hspace{0.5cm} \textrm{where} \hspace{0.5cm}}
\newcommand{\hgap}{\hspace{0.5cm}}
\newcommand{\pfrac}[2]{\frac{\partial #1}{\partial #2}}
\newcommand{\doubleformula}[4]{\begin{center} \begin{tcolorbox}[title = #2] $$#3$$\\$$#4$$\end{tcolorbox}\end{center}}
\newcommand{\curly}[1]{\left\{#1\right\}}
\newcommand{\proj}[2]{{}\textrm{proj}_{#1}\left(#2\right)}
\newcommand{\sgap}{\smallskip\\}

\newcommand{\ds}{\displaystyle}
\newcommand{\Arg}{\textrm{Arg}}
\newcommand{\tabitem}{~~\llap{\textbullet}~~}


\newcommand{\tripleformula}[5]{\begin{center} \begin{tcolorbox}[title = #2] $$#3$$\\$$#4$$\\$$#5$$\end{tcolorbox}\end{center}}

\usepackage{physics}
% \usepackage{braket}

\begin{document}
\begin{center}
    \Large PHYS 401 Radiation Notes\\
    \normalsize Reese Critchlow
\end{center}
    
We can separate the calculations for radiation in three main ways:
\gap
\header{Ideal Electric Dipole}
For an ideal electric dipole in the $\hat{z}$ direction, $\vec{p}\,(t) = p_0\cos(\omega t)\hat{\pmb{z}}$ we get that:
\begin{align*}
    V(r, \theta, t) &\approx -\frac{p_0 \omega}{4\pi \epsilon_0c}\left(\frac{\cos \theta}{r}\right) \sin \left[\omega(t- r/c)\right]
    && \mathbf{A}(r, \theta, t) \approx -\frac{\mu_0 p_0 \omega}{4\pi r} \sin \left[\omega(t - r/c)\right]\hat{\pmb{z}}\\
    \mathbf{E}(\vec{r}, t) &\approx -\frac{\mu_0 p_0 \omega^2}{4\pi}\left(\frac{\sin\theta}{r}\right)\cos\left[\omega(t- r/c)\right] \hat{\pmb{\theta}}
    && \mathbf{B}(\vec{r}, t) = - \frac{\mu_0 p_0 \omega^2}{4\pi c}\left(\frac{\sin\theta}{r}\right)\cos\left[\omega(t- r/c)\right]\hat{\pmb{\phi}}\\
    \mathbf{S}(\vec{r}, t) &\approx \frac{\mu_0}{c}\curly{\frac{p_0\omega^2}{4\pi} \left(\frac{\sin\theta}{r}\right)\cos[\omega(t-r/c)]}^2 \hat{\pmb{r}}
\end{align*}

We can attempt to generalize this in more ``general coordinates'':
\begin{align*}
    V(\vec{r}, t) &\approx \frac{1}{4\pi \epsilon} \frac{\vec{n} \cdot \dot{\vec{p}}\,(t_0)}{rc} && \vec{A}(\vec{r}, t) \approx \frac{\mu_0 }{4\pi} \frac{\dot{\vec{p}}\,(t_0)}{r}\\
    \vec{E}(\vec{r}, t) &\approx \frac{\mu_0}{4\pi r}\left[\vec{n} \times (\vec{n} \times \ddot{\vec{p}}\,)\right]&&
    \vec{B}(\vec{r}, t) \approx -\frac{\mu_0}{4\pi rc}\left(\vec{n} \times \ddot{\vec{p}}\right)\\
    \vec{S}(\vec{r}, t) &\approx \frac{\mathbf{E} \times \mathbf{B}}{\mu_0} = \frac{\mu_0 \vec{n}}{16\pi^2 r^2c^2}\left[\ddot{\vec{p}}\,^2 - (\vec{n} \cdot \ddot{\vec{p}}\,)^2\right]
\end{align*}
\gap 
For all dipole orientations, one can also use the following:
\begin{align*}
    \langle \mathbf{S} \rangle &= \left(\frac{\mu_0 p_0^2 \omega^4}{32 \pi^2 c}\right)\frac{\sin^2\theta}{r^2}\pmb{\hat{r}}\\
    \langle P \rangle &= \int \langle \mathbf{S} \rangle \cdot d \mathbf{a} = \frac{\mu_0 p_0^2 \omega^4}{12 \pi c^3}
\end{align*}

\header{General Radiation Formulae}
Recall the formula for generalized dipole moment:
\begin{align*}
    \mathbf{p}(t) = \int \mathbf{r}' \rho(\mathbf{r}', t_0) d^3 \mathbf{r}'
\end{align*}

Given this, we can use the generalized coordinates formula from the prior section.
\gap
This also extends to the \underline{total radiated power}:
\begin{align*}
    P_\textrm{rad}(t_0) \approxeq \frac{\mu_0}{6\pi c}\left[\ddot{p}(t_0)\right]^2
\end{align*}

Note that this power $P$ is energy per unit time, so we can also write $P = \frac{dW}{dt}$, where $W$ is energy.
\gap 
We also introduce the Larmor Formula, which gives the power of an arbitrary moving charge 
with acceleration $a(t)$:
\begin{align*}
    P = \frac{\mu_0q^2}{6\pi c}[a(t)]^2
\end{align*}
\pagebreak

\header{Magnetic Dipole Radiation}
Given some magnetic dipole formed from some oscillating current $I(t) = I_0 \cos(\omega t)$ in a wire loop fo radius $b$,
the dipole takes the form:
\begin{align*}
    \mathbf{m}(t) = \pi b^2 I(t)\pmb{\hat{z}} = m_0 \cos(\omega t)\pmb{\hat{z}}
\end{align*}
Thus, we get some similar equations:
\begin{align*}
    \mathbf{A}(r, \theta, t) &\approx - \frac{\mu_0 m_0 \omega}{4\pi c}\left(\frac{\sin\theta}{r}\right)\sin \left[\omega(t-r/c)\right]\pmb{\hat{\phi}}\\
    \mathbf{E}(\mathbf{r}, t) &\approx \frac{\mu_0 m_0 \omega^2}{4\pi c} \left(\frac{\sin \theta}{r}\right) \cos\left[\omega(t- r/c)\right]\pmb{\hat{\phi}} &&
    \mathbf{B}(\mathbf{r}, t) \approx \frac{-\mu_0 m_0 \omega^2}{4\pi c^2}\left(\frac{\sin \theta}{r}\right) \cos \left[\omega(t- r/c)\right] \pmb{\hat{\theta}}
\end{align*}

We can actually generalize this as in the same way to the electric case:
\begin{align*}
    V &= 0\\
    \mathbf{A}(\mathbf{r}, t) &\approx \frac{\mu_0}{4\pi}\frac{\vec{m} \times \vec{n}}{r^2} = \frac{\mu_0}{4\pi} \frac{\vec{m} \times \vec{r}}{r^3}\\
    \mathbf{E}(\mathbf{r}, t) &\approx - \frac{\mu_0}{4\pi rc} (\vec{n} \times \ddot{\vec{m}}) \\
    \mathbf{B}(\mathbf{r}, t) &\approx \frac{\mu_0}{4\pi rc^2} \left[\vec{n} \times (\vec{n} \times \ddot{\vec{m}})\right]\\
    \mathbf{S}(\mathbf{r}, t) &\approx \frac{\mu_0\vec{n}}{16\pi^2 r^2 c^3}\left[\ddot{\vec{m}}\,^2 - (\vec{n} \cdot \ddot{\vec{m}}\,)^2\right]
\end{align*}

\begin{align*}
    \omega = kv
\end{align*}

\end{document}
      
