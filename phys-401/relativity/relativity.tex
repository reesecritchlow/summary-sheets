\documentclass{article}
\usepackage[utf8]{inputenc}
\usepackage[letterpaper, portrait, margin=1in]{geometry}
\usepackage{multicol}
\usepackage{amsmath}
\usepackage{amssymb}
\usepackage{enumerate}
\setlength\parindent{0pt}
\usepackage{enumerate}
\usepackage{graphicx}
\graphicspath{ {./images/} }
\usepackage{fancyhdr}
\usepackage{tcolorbox}
\hyphenchar\font=-1
\usepackage{tabularx}

\newcommand{\header}[1]{\begin{large}\noindent #1\end{large}\\\rule{\textwidth}{0.5pt}}
\newcommand{\sheader}[1]{\underline{#1:}}
\newcommand{\mheader}[1]{\textbf{{\sheader{#1}}}}
\newcommand{\gap}{\medskip\\}
\newcommand{\centertext}[1]{\begin{center}#1\end{center}}
\newcommand{\bfrac}[2]{\left(\frac{#1}{#2}\right)}
\newcommand{\formula}[3]{\begin{center} \begin{tcolorbox}[title = #2] $$#3$$\end{tcolorbox}\end{center}}
\newcommand{\where}{\hspace{0.5cm} \textrm{where} \hspace{0.5cm}}
\newcommand{\hgap}{\hspace{0.5cm}}
\newcommand{\pfrac}[2]{\frac{\partial #1}{\partial #2}}
\newcommand{\doubleformula}[4]{\begin{center} \begin{tcolorbox}[title = #2] $$#3$$\\$$#4$$\end{tcolorbox}\end{center}}
\newcommand{\curly}[1]{\left\{#1\right\}}
\newcommand{\proj}[2]{{}\textrm{proj}_{#1}\left(#2\right)}
\newcommand{\sgap}{\smallskip\\}

\newcommand{\ds}{\displaystyle}
\newcommand{\Arg}{\textrm{Arg}}
\newcommand{\tabitem}{~~\llap{\textbullet}~~}


\newcommand{\tripleformula}[5]{\begin{center} \begin{tcolorbox}[title = #2] $$#3$$\\$$#4$$\\$$#5$$\end{tcolorbox}\end{center}}

\usepackage{physics}
% \usepackage{braket}

\begin{document}

\begin{center}
        \Large PHYS 401 Relativity Notes\\
        \normalsize Reese Critchlow
\end{center}

\header{Lorentz Transformations}

For any frame $\mathcal{S}'$ travelling at velocity $v$ relative to some other
frame $\mathcal{S}$ along the $x$-axis, the new coordinates of the new frame $\mathcal{S}'$
are given by the following:
\begin{align*}
    t' &= \frac{t- \frac{v}{c^2}x}{\sqrt{1- \frac{v^2}{c^2}}}\\
    x' &= \frac{x- vt}{\sqrt{1 - \frac{v^2}{c^2}}}\\
    y' &= y\\
    z' &= z
\end{align*}

We can then also derive the inverse Lorentz transformations:
\begin{align*}
    t &= \gamma \left(t' + \frac{v}{c^2} x'\right)\\
    x &= \gamma (x' + vt')\\
    y &= y'\\
    z &= z'\\
\end{align*}

\underline{Lorentz Invariance} are any quantities that remain the same under a Lorentz transformation. These include:
\begin{itemize}
    \item The speed of light.
    \item The spacetime interval between any two points in Minkowski space.
    \item The scalar product of any two four-vectors.
    \item $\vec{E} \cdot \vec{B}$ is invariant.
    \item $\frac{1}{c^2}\vec{E} - \vec{B}$
\end{itemize}

Where $\ds \gamma = \frac{1}{\sqrt{1 - \frac{v^2}{c^2}}}$.
\gap
This can also be represented using matricies:
\begin{align*}
    \begin{pmatrix}
        ct'\\
        x_1'\\
        x_2'\\
        x_3'\\
    \end{pmatrix}
    = \begin{pmatrix}
        \gamma & - \gamma \beta & 0 & 0\\
        -\gamma \beta & \gamma & 0 & 0\\
        0 & 0 & 1 & 0\\
        0 & 0 & 0 & 1\\
    \end{pmatrix}\begin{pmatrix}
        ct\\x_1\\x_2\\x_3
    \end{pmatrix}
\end{align*}
\header{4-Vectors}
In the notion of Minkowski space, we have that each point in the 
universe has a position and a time. Hence, like in simple positional/3-space,
we can define a 4-vector to capture both space and time.

\begin{align*}
    A_\mu = \mathbf{A} &= (A_t, A_x, A_y, A_z)\\
    &= (A_t, A_i)\\
    &= (a_t, \mathbf{a})
\end{align*}

Thus, we define the dot product for a four-vector to be the following:
\begin{align*}
    a_\mu b_\mu &= a_tb_t - \mathbf{a} \cdot \mathbf{b}.
\end{align*}

This dot product makes it such that the 4-vector is invariant under 
rotations or frame changes.

\sheader{Common 4-vectors}

\renewcommand{\arraystretch}{1.5}
\begin{tabular}{|c|c|}
    \hline
    \underline{Quantity} & \underline{Form} \\
    \hline
    4-velocity & $\ds X_\mu = (ct, \vec{x}) = (ct, x, y, z)$ \\
    \hline
    4-velocity & $\ds \gamma(c, \vec{u}) = \gamma\left(c, \frac{dx}{dt}, \frac{dy}{dt}, \frac{dz}{dt}\right)$ \\
    \hline
    4-momentum & $\ds P_\mu = (\gamma m c, \gamma m \vec{v}) = \left(\frac{E}{c}, \vec{p}\right) = \left(\frac{E}{c}, p_x, p_y, p_z\right)$\\
    \hline
    4-current & $\ds J_\mu = (c\rho, \vec{j}) = (c\rho, j_x, j_y, j_z)$ \\
    \hline
    4-potential & $\ds \left(\frac{V}{c}, \vec{A}\right) = \left(\frac{V}{c}, A_x, A_y, A_z\right)$\\
    \hline
\end{tabular}
\gap
\sheader{Four-Dimensional Gradient}

We can define the four-dimensional gradient to be the following:

\begin{align*}
    \nabla_\mu = \left(\frac{\partial}{\partial t}, -\pmb{\nabla} \right),
\end{align*}

which, when applied to a 4-vector yields the following:
\begin{align*}
    \nabla_\mu a_\mu = \frac{\partial}{\partial t} a_t + \pmb{\nabla} \cdot \mathbf{a}.
\end{align*}

\renewcommand{\arraystretch}{3}
\begin{tabular}{|c|c|}
    \hline
    \underline{Derivative} & \underline{Form} \\
    \hline
    4-divergence & $\ds \nabla_\mu a_\mu = \frac{\partial a_t}{\partial t} + \pmb{\nabla} \cdot{a}$ \\
    \hline
    4-gradient & $\ds \nabla_\mu \lambda = \left(\frac{\partial \lambda}{\partial t}, - \pmb{\nabla} \lambda\right)$ \\
    \hline
    4-laplacian (d'Alembertian) & $\ds \nabla_\mu \nabla_\mu = \frac{\partial^2}{\partial t^2} - \nabla^2 = \Box^2$\\
    \hline
\end{tabular}

\vspace{1cm}
\header{Applying 4-Dimensional Quantities to Electrodynamics}

As we knew from the 3-dimensional case:

\begin{align*}
    \Box^2 V = \frac{\rho}{\epsilon_0} && \Box^2 \mathbf{A} = \frac{\mathbf{J}}{\epsilon_0}.
\end{align*}

Hence, we can generalize to get the following form:

\begin{align}
    \Box^2 A_\mu = \frac{j_\mu}{\epsilon_0}
\end{align}

We can also say that these equations hold only if the Lorentz 
gauge is obeyed. This is called the \underline{Lorenz Condition} and 
can be written $\nabla_\mu A_\mu = 0$. The Lorenz Condition is said 
to be an invariant condition and thus, for all frames (1) holds.
\gap

We can also say that the charage conservation equation can be 
written in similar terms:
\begin{align*}
    \nabla_\mu j_\mu = \frac{\partial \rho}{\partial t} + \pmb{\nabla} \cdot \mathbf{J} = 0
\end{align*}
and that gauge invariance has that:
\begin{align*}
    A_\mu' = A_\mu - \nabla_\mu f
\end{align*}
Maxwell's equations can also be rewritten in terms of four-vectors:
\begin{align*}
    \Box^2 A_\mu = j_\mu \mu_0
\end{align*}
We can also look at the lorentz transformations for $V$ and $A_x$:\\
\textit{Note: these give $V'$ and $A_x'$ in the \underline{unprimed} coordinates, not the primed ones.}
\begin{align*}
    \frac{V'}{c} = \gamma \left(\frac{V}{c} - \beta A_x\right) && A_x' = \gamma\left(A_x - \beta \frac{V}{c}\right)
\end{align*}
For a moving point charge in the $x$ direction, we can say that:
\begin{align*}
    V &= \frac{q}{4\pi \epsilon_0} \frac{1}{\sqrt{(x-vt)^2 + \left(1 - \frac{v^2}{c^2}\right) (y^2 + z^2)}}\\
    A_x &=\frac{v}{c^2}\frac{q}{4\pi \epsilon_0} \frac{1}{\sqrt{(x-vt)^2 + \left(1 - \frac{v^2}{c^2}\right) (y^2 + z^2)}}\\
    \vec{E} &= \frac{q}{4\pi \epsilon_0}\left(1 - \frac{v^2}{c^2}\right) \left[
        \frac{(x-vt) \hat{x} + y\hat{y} + z \hat{z}}
        {\left[(x-vt)^2 + \left(1- \frac{v^2}{c^2}\right)(y^2 + z^2)\right]^{\frac{3}{2}}}
    \right]
    \\
    \vec{B} &= \frac{\vec{v} \times \vec{E}}{c^2}\\
    \vec{S} &= \frac{-\vec{E} (\vec{E} \cdot \vec{v}\,) + \vec{v}\vec{E}\,^2}{\mu_0 c^2}
\end{align*}

\sheader{Field Tensor} Since we cannot possibly contain all of the 
information of the fields within a simple 4-vector, we use a tensor to 
describe the fields. This tensor $F$ is an antisymmetric tensor 
such that $-F = F$. Each entry in $F$ is given by:
\begin{align*}
    F_{\mu\nu} = \nabla_\mu A_\nu - \nabla_\nu A_\mu
\end{align*}
and thus, we can define the tensor as:
\begin{align*}
    F = 
    \renewcommand{\arraystretch}{1.25}
    \begin{bmatrix}
        0 & -E_x / c & -E_y/c & -E_z/c \\
        E_x/c & 0 & -B_z & B_y\\
        E_y/c & B_z & 0 & -B_x\\
        E_z/c & -B_y & B_x & 0
    \end{bmatrix} = 
    \begin{bmatrix}
        0 & F_{0x} & F_{0y} & F_{0z} \\
        F_{x0} & 0 & F_{xy} & F_{xz}\\
        F_{y0} & F_{yx} & 0 & F{yz}\\
        F_{z0} & F_{zx} & F_{zy} & 0
    \end{bmatrix}
\end{align*}
\pagebreak

\sheader{Generalized Relativistic Transformations}

Generalizing this, we can write the transforms for $\mathbf{E}$ and $\mathbf{B}$ fields in a generalized sense, given some 
$\mathcal{S}'$ frame with velocity $v_x$ relative to some $\mathcal{S}$ frame.
Again, these all give results in the \underline{\textit{unprimed}} coordinates.
\begin{align*}
    E_x' &= E_x && B_x' = B_x\\
    E_y' &= \gamma(E_y - v B_z) && B_y' = \gamma\left(B_y + \frac{v}{c^2} E_z\right)\\
    E_z' &= \gamma(E_z + vB_y) && B_z' = \gamma\left(B_z - \frac{v}{c^2} E_y\right)
\end{align*}

We can generalize this further using the following forms:
\begin{align*}
    E_x' &= E_x && B_x' = B_x\\
    E_y' &= \gamma (\mathbf{E} + \mathbf{v} \times \mathbf{B})_y && B_y' = \gamma (\mathbf{B} - \mathbf{v} \times \mathbf{E})_y\\
    E_z' &= \gamma (\mathbf{E} + \mathbf{v} \times \mathbf{B})_z && B_y' = \gamma (\mathbf{B} - \mathbf{v} \times \mathbf{E})_z
\end{align*}

\sheader{4-Force}

\begin{align*}
    f_\mu = \left(\frac{\mathbf{F} \cdot \frac{\mathbf{v}}{c}}{\sqrt{1 - \frac{v^2}{c^2}}}, \frac{\mathbf{F}}{\sqrt{1 - \frac{v^2}{c^2}}}\right)
\end{align*}

Can be generalized:

\begin{align*}
    ds = dt \sqrt{1 - \frac{v^2}{c^2}}
\end{align*}
and we can rewrite as:
\begin{align*}
    \frac{dx_\mu}{ds} = u_\mu
\end{align*}
as well as:
\begin{align*}
    \frac{dp_\mu}{ds}= f_\mu
\end{align*}
Or, lastly:
\begin{align*}
    m_0 \frac{d^2x_\mu}{ds^2} = f_\mu = qu_\nu F_{\mu \nu}
\end{align*}

\sheader{Generalizing, even further}

For some:
\begin{align*}
    F_{\mu\nu} = a_\mu b_\nu - a_\nu b_\mu
\end{align*}

We get that:
\begin{align*}
    a_0' &= \frac{a_0 - \frac{v}{c} a_x}{\sqrt{1 - \frac{v^2}{c^2}}} && b_0' = \frac{b_0 - \frac{v}{c} b_x}{\sqrt{1 - \frac{v^2}{c^2}}}\\
    a_x' &= \frac{a_x - \frac{v}{c} a_0}{\sqrt{1 - \frac{v^2}{c^2}}} && b_x' = \frac{b_x - \frac{v}{c} b_0}{\sqrt{1 - \frac{v^2}{c^2}}}\\
    a_{y, z}' &= a_{y, z}  && b_{y, z}' = b_{y, z}
\end{align*}

\end{document}
