\documentclass{article}
\usepackage[utf8]{inputenc}
\usepackage[letterpaper, portrait, margin=1in]{geometry}
\usepackage{multicol}
\usepackage{amsmath}
\usepackage{amssymb}
\usepackage{enumerate}
\setlength\parindent{0pt}
\usepackage{enumerate}
\usepackage{graphicx}
\graphicspath{ {./images/} }
\usepackage{fancyhdr}
\usepackage{tcolorbox}

\usepackage{calligra}
\DeclareMathAlphabet{\mathcalligra}{T1}{calligra}{m}{n}
\DeclareFontShape{T1}{calligra}{m}{n}{<->s*[2.2]callig15}{}

\newcommand{\sepvec}{\vec{r_\textrm{sep}}}

\newcommand{\kfrac}{\frac{1}{4\pi\epsilon_0}}

\newcommand{\scripty}[1]{\ensuremath{\mathcalligra{#1}}}


\newcommand{\header}[1]{\begin{large}\noindent #1\end{large}\\\rule{\textwidth}{0.5pt}}
\newcommand{\gap}{\medskip\\}
\newcommand{\centertext}[1]{\begin{center}#1\end{center}}
\newcommand{\bfrac}[2]{\left(\frac{#1}{#2}\right)}
\newcommand{\formula}[3]{\begin{center} \begin{tcolorbox}[title = #2] $$#3$$\end{tcolorbox}\end{center}}
\newcommand{\where}{\hspace{0.5cm} \textrm{where} \hspace{0.5cm}}
\newcommand{\hgap}{\hspace{0.5cm}}
\newcommand{\pfrac}[2]{\frac{\partial #1}{\partial #2}}
\newcommand{\jomega}{{j\omega}}
\newcommand{\domega}{d\omega}

\newcommand{\sheader}[1]{\underline{#1:}}
\newcommand{\smallgap}{\smallskip\\}
\newcommand{\sgap}{\smallskip\\}

\newcommand{\doubleformula}[4]{\begin{center} \begin{tcolorbox}[title = #2] $$#3$$\\$$#4$$\end{tcolorbox}\end{center}}

\newcommand{\tripleformula}[5]{\begin{center} \begin{tcolorbox}[title = #2] $$#3$$\\$$#4$$\\$$#5$$\end{tcolorbox}\end{center}}

\title{PHYS 301 Notes}
\author{reesecritchlow }
\date{September 2022}

\begin{document}

\begin{center}
    \Large PHYS 301 Notes\\
    \normalsize Reese Critchlow
\end{center}

\section*{Midterm 1}

\header{Calculus Review}

\sheader{Spherical Coordinates}
\smallskip\\
Sometimes, spherical coordinates are easier to work with than cartesian coordinates.
When using spherical coordinates, there are some key things to note:
\begin{enumerate}
    \item \sheader{The coordinates themselves:} 
    \smallgap
    $\phi$: Equatorial Azimuth $[0, 2\pi]$ (from the ``$x$'' axis)\\
    $\theta$: Axial Azimuth $[0, \pi]$ (from the ``$z$'' axis)\\
    $r$: Radial Distance
    \item \underline{The infinitesimal displacement is different:}
    \smallgap
    $d\vec{l} = dr \, \hat{r} + r d\theta \, \hat{\theta} + r \sin \theta d\phi \, \hat{\phi}$
    \item \underline{The volume element is different:}
    \smallgap
    $d\tau = r^2 \sin \theta \, dr \, d\theta \, d\phi$
\end{enumerate}

\sheader{Cylindrical Coordinates:} See textbook, this would be too redundant.
\gap
\sheader{Rules for Irrotational Fields/Conservative Fields}
\begin{enumerate}
    \item $\vec{\nabla} \times \vec{F} = \vec{0}$ everywhere.
    \item $\int_a^b\vec{F}\cdot d\vec{l}$ is independent of path, for any given end points.
    \item $\oint \vec{F}\cdot d\vec{l} = 0$ for any closed loop.
    \item $\vec{F}$ is the gradient of some scalar function: $\vec{F} = -\nabla V$.
\end{enumerate}
It is important to note that since $\vec{E} = -\nabla V$, then all electrostatic fields 
are irrotational.
\sgap
\sheader{Rules for Divergence-less fields.}
\begin{enumerate}
    \item $\vec{\nabla} \cdot \vec{F} = 0$ everywhere.
    \item $\int \vec{F} \cdot d\vec{a}$ is independent of surface, for any given boundary line.
    \item $\oint \vec{F} \cdot d\vec{a} = 0$ for any closed surface.
    \item $\vec{F}$ is the curl of some vector function.
\end{enumerate}
\sheader{Taylor Expansions} Often, it is difficult to obtain limits for when one quantity
gets significantly larger than another in equations describing fields or potentials.
Thus, the \underline{Taylor Expansion} is helpful for this. Take for example a quantity $a$
and a quantity $b$. For $b >> a$, it is useful to try to bring the equation into a form
such that $\left(\frac{a}{b}\right)^n, n \in \mathbb{N}$. Since $b >> a$, then we can say that $\frac{a}{b} \approx 0$
Thus, a Taylor series centered around $x = 0$ is obtained, which can be evaluated with
the expression:
\[
    f(x) \approx f(0) + f'(0) \cdot x  + f''(0) \cdot \frac{x^2}{2} + \cdots + \frac{f^{(n)}x^n}{n!}   
\]
Generally, it is best practice to evaluate the Taylor series up until the degree of the
largest polynomial in the original function.

\pagebreak
\header{Electric Fields}

It is important to recognize the notation in the Griffith's textbook, which is used
primarily for this course when working with Fields and directions:
\sgap
$\vec{r}$: distance from the origin to a ``field point''.\\
$\vec{r'}$: distance from the origin to the charge\\
$\sepvec$ : distance from the charge to the ``field point''\\
Thus, it is given that $\sepvec = \vec{r} - \vec{r'}$.
\gap
Now, while working with Coulomb's law, we can define the electric field due to a point charge to be:
\[
    \vec{E} = \frac{1}{4\pi \epsilon_0} \frac{q}{||\sepvec||^2}\sepvec    
\]
\sheader{Types of Field Integrations} There exist three main types of field integrations:
\begin{multicols}{3}
    \centertext{\underline{Line Charge}}
    \[
        \vec{E}(\vec{r}) = \frac{1}{4 \pi \epsilon_0}\int{\frac{\lambda(\vec{r'})}{||\sepvec||^2}\sepvec \, dl'}    
    \]
    \vfill\null\columnbreak
    \centertext{\underline{Surface Charge}}
    \[
        \vec{E}(\vec{r}) = \frac{1}{4\pi\epsilon_0}\int{\frac{\sigma(\vec{r'})}{||\sepvec||^2}\sepvec \, da'}    
    \]
    \vfill\null\columnbreak
    \centertext{\underline{Volume Charge}}
    \[
        \vec{E}(\vec{r}) = \frac{1}{4\pi\epsilon_0}\int{\frac{\rho(\vec{r'})}{||\sepvec||^2}\sepvec \, d\tau'}    
    \]
    \vfill\null
\end{multicols}
\sheader{Gauss's Law} Gauss's law is as follows:
\begin{align*}
    \oint{\vec{E} \cdot d\vec{a}} = \frac{Q_\textrm{encl}}{\epsilon_0} && \vec{\nabla} \cdot \vec{E} = \frac{\rho}{\epsilon_0}
\end{align*}
By knowing the geometry and symmetry of a situation however, Gauss's law can require
no integration whatsoever. This is when we know that field is the same for every point
on a surface. 
\gap
\header{Electric Potential}
The potential of a point in a field relative to another is defined as:
\[
    V(\vec{r}) \equiv - \int_{\mathcal{O}}^r \vec{E} \cdot d\vec{l}    
\]
Which is often referred to as the potential difference between two points:
\[
    V(\vec{b}) - V(\vec{a}) = - \int_{\vec{a}}^{\vec{b}}{\vec{E} \cdot d\vec{l}}
\]
It is also to be noted the relation between field and potential:
\[
    \vec{E} = - \nabla V    
\]
\sheader{Remark} it is important to note that unlike electric field, electric potential
is a \underline{scalar}. It has no direction, and should be handled accordingly.
\gap
Potential can also be obtained for a volume charge:
\[
    V(\vec{r}) = \kfrac \int{\frac{\rho(\vec{r'})}{||\sepvec||}d\tau'}
\]
Similar formulae can be extrapolated for line and surface charge distributions, analogous
to those for electric fields.
\pagebreak
\gap
\header{Work and Energy in Electrostatics}
To calculate the work that it takes to get from one point to another in a field,
we can use a \underline{work integral}. We can also employ the potential difference.
\[
    W = \int_{\textbf{a}}^{\textbf{b}} \vec{F} \cdot d\vec{l} = -Q\int_{\textbf{a}}^{\textbf{b}}{\vec{E} \cdot d\vec{l}} = Q[V(\vec{b}) - Q(\vec{a})]
\]
It is important to note that for the second form of this integral uses \textbf{negative} $Q$,
not positive. This is because the integral by default describes the amount of work that
the \underline{field does}, not the work that \underline{is required}. One can use 
a positive value of $Q$ if trying to find out how much work the field does.
\gap
It is also to be noted that the potential of a system is the work that is required
to create the system \underline{per unit charge}.
\gap
\sheader{Work and Point Charges} We can also describe the amount of work that it takes
to assemble a collection of point charges:
\[
    W = \frac{1}{2}\sum_{i = 1}^n{q_i V(\vec{r_i})}    
\]
\sheader{Work of a Continuous Charge Distribution} To describe the amount of energy
that a charge distribution has, or the amount of energy required to create it in 
empty space, the following formula can be used:
\[
    W = \frac{\epsilon_0}{2}\int\limits_{\textrm{all space}}E^2 d\tau    
\]

\header{Boundary Conditions}
One of the important equations in the relationships between $\rho$, $V$, and $E$ is 
$\nabla^2 V = - \frac{\rho}{\epsilon_0}$. However, this equation often leaves one 
with a number of undetermined coefficients from the integration. Thus, it is important
to investigate the boundary conditions of fields and potentials.
\gap
From this, a set of important boundary conditions emerges for a boundary at $a$.
\begin{itemize}
    \item \sheader{Continuity of Potential} Since $\nabla V = -\vec{E}$, it is known 
    that potential should be continuous along all space. Thus,
    \[
        V_\textrm{above}(a) = V_\textrm{below}(a)
    \]
    \item \sheader{Preservation of Field} For any field interacting with a surface charge,
    by Gauss's law, it is known that the difference between the field above and the field
    below is simply the surface charge density over the permittivity of free space:
    \[
        \vec{E}_\textrm{above} - \vec{E}_\textrm{below} = \frac{\sigma}{\epsilon_0}\hat{n}    
    \]
    Where $\hat{n}$ is the normal vector of the boundary surface. The presence of the
    normal vector is important, because it requires the boundary surface to have 
    symmetry with the field above and the field below.
    \item \sheader{Bounding of Potential (not general)} Since it is known that potential should generally 
    be bounded over all space (except in some cases), it is appropriate to say that
    \[
        V(0) \textrm{ is bounded.}    
    \]
    This is generally more true for configurations with spherical symmetry.
    \item \sheader{Zero Potential at Infinity (not general)} For many spherically-symmetric
    examples, it is known that the potential at infinity is zero. Thus, it can be said that:
    \[
        V(\infty) = 0.    
    \]
\end{itemize}
\pagebreak

\header{Uniqueness of Solutions}
Given the relationship $\nabla^2 V = -\frac{\rho}{\epsilon_0}$, there exists the case
that $\rho =0$, for which a distinct set of attributes arise:
\begin{enumerate}
    \item $V$ has no local maxima nor minima inside. The maxima and minima are located
    on the surrounding boundaries.
    \item $V$ is smooth and continuous, everywhere.
    \item $V(\vec{r})$ is the average of $V$ over surface of any surrounding sphere:
    $V(\vec{r}) = \frac{1}{4\pi R^2}\oint VdA$.
    \item $V$ is unique, as the solution of the Laplace equation is uniquely determined
    if $V$ is specified on the boundary surface around the volume.
\end{enumerate}


\end{document}