\documentclass{article}
\usepackage[utf8]{inputenc}
\usepackage[letterpaper, portrait, margin=1in]{geometry}
\usepackage{multicol}
\usepackage{amsmath}
\usepackage{amssymb}
\usepackage{enumerate}
\setlength\parindent{0pt}
\usepackage{enumerate}
\usepackage{graphicx}
\graphicspath{ {./images/} }
\usepackage{fancyhdr}
\usepackage{tcolorbox}


\newcommand{\header}[1]{\begin{large}\noindent #1\end{large}\\\rule{\textwidth}{0.5pt}}
\newcommand{\gap}{\medskip\\}
\newcommand{\centertext}[1]{\begin{center}#1\end{center}}
\newcommand{\bfrac}[2]{\left(\frac{#1}{#2}\right)}
\newcommand{\formula}[3]{\begin{center} \begin{tcolorbox}[title = #2] $$#3$$\end{tcolorbox}\end{center}}
\newcommand{\where}{\hspace{0.5cm} \textrm{where} \hspace{0.5cm}}
\newcommand{\hgap}{\hspace{0.5cm}}
\newcommand{\pfrac}[2]{\frac{\partial #1}{\partial #2}}
\newcommand{\jomega}{{j\omega}}
\newcommand{\domega}{d\omega}

\newcommand{\doubleformula}[4]{\begin{center} \begin{tcolorbox}[title = #2] $$#3$$\\$$#4$$\end{tcolorbox}\end{center}}

\newcommand{\tripleformula}[5]{\begin{center} \begin{tcolorbox}[title = #2] $$#3$$\\$$#4$$\\$$#5$$\end{tcolorbox}\end{center}}

\title{ELEC 221 Notes}
\author{reesecritchlow }
\date{September 2022}

\begin{document}

\begin{center}
    \Large ELEC 221 Notes\\
    \normalsize Reese Critchlow
\end{center}

\header{Notation}

\begin{align*}
    \textrm{Continuous Signals} && \textrm{Discrete Signals}\\
    x(t) && x(n)\\
    \textrm{Continuous-Time Systems} && \textrm{Discrete-Time Systems}\\
    x(t) \to y(t) && x[n] \to y[n]
\end{align*}
The arrow for systems means that for a given signal, say $x(t)$ when in putted into a system ($\to$) yields an output signal $y(t)$.
\\\gap
An equals sign ($=$) simply indicates equality between two signals.
\\\gap
\header{System Properties}
\begin{enumerate}
    \item \textbf{Memory}\\
    A system is \underline{memoryless} if the output at each time depends only on the input at the same time.\\
    Examples:
    \begin{align*}
        \textrm{Example: Voltage Through a Resistor} && V(t) = RI(t)\\
        \textrm {Counter Example: Voltage through a Capacitor} && V(t) = \frac{1}{C}\int_{-\infty}^tI(\tau)d\tau \\
        \textrm{Counter Example: A Delay System} && y[n] = x[n-1]
    \end{align*}
    Generally, a system where the function has some sort of time depenence which is not in the instantaneous current time period has a memory.
    
    \item \textbf{Invertibility}\\
    A system is \underline{invertible} if distinct inputs lead to distinct outputs.\\
    An example of a noninvertible system is $y(t) = x^2(t)$, because the square destroys a negative sign that may have existed in the input signal.
    
    \item \textbf{Causality}\\
    A system is \underline{causal} if the output at \textit{any time} depends only on the input at the present time or in the past. A comparison between causal systems and systems with memories may be able to be drawn, but they are not the same.
    \begin{multicols}{2}
        \underline{Causal Systems}
        \begin{itemize}
            \item Dependent events can only be in the present or in the past.
        \end{itemize}
        \vfill\null\columnbreak
        \underline{Memory-Bearing Systems}
        \begin{itemize}
            \item Dependent events can be in the present or in the future.
        \end{itemize}
        \vfill\null
    \end{multicols}
    A capacitor is causal, but a moving average, or time reversal is not.
    \item \textbf{Stability}\\
    A system is \underline{stable} if small changes in input do not cause the output to diverge.\\
    A stable system can also be described as one where bounded inputs lead to bounded outputs; essentially that the system never reaches an unbounded value.\\
    A system is also stable if \underline{the impulse response is absolutley integrable}.
    \item \textbf{Time Invariance}\\
    A system is \underline{time invariant} if time shifts in the input lead to identical time shifts in the output.\\
    \gap
    \underline{Proving Time Invariance}: Shift the input signal by a time period $a$, and pass it through the system. Shift the un-shifted output signal of the system by the same time period $a$. If they are the same signal, the system is time invariant.
    \item \textbf{Linearity}
    A \underline{linear} system has the following properties:
    
    \begin{center}
        \begin{multicols}{2}
            \underline{Additivity}
            \[
                x_1(t) + x_2(t) \to y_1(t) + y_2(t)
            \]
            \vfill\null\columnbreak
            \underline{Homogeneity}
            \[
                ax_1(t) \to ay_1(t)
            \]
            \vfill\null
        \end{multicols}
    \end{center}
    
    These two properties can also be combined:
    \[
        ax_1(t) + bx_2(t) \to ay_1(t) + by_2(t)
    \]
    
\end{enumerate}

\header{Harmonics}
A signal that has \textbf{only odd harmonics} requires that:
\[
    f\left(t+\frac{T}{2}\right) = -f(t)
\]
Whereas a signal that has \textbf{only even harmonics} requires that:
\[
    f\left(t + \frac{T}{2}\right) = f(t)
\]

\header{Impulse Response}

The \textbf{Impulse Response} of a system is the signal a system produces after a
unit impulse signal is passed through it. The unit impulse response is denoted by:
\begin{align*}
     h(n) && h[n].
\end{align*}
The impulse response is generally used to determine the ouput of a system 
by means of the convolution.
\gap
\header{Convolution}

The \textbf{Convolution} operation multiplies the entries of one signal by another in
a systematic fashion. It is defined by:
\begin{align*}
    x(t) \ast h(t) = \int_{-\infty}^\infty x(\tau) h(t- \tau) {d}\tau && x[n] \ast h[n] = \sum_{k=-\infty}^\infty x[k]h[n-k]    
\end{align*}
In this context, the convolution operator is being used to determine the output of a
signal through a system, given the system's impulse response.
\gap
\underline{Properties of the Convolution:} The convolution is:
\begin{itemize}
    \item Associative: $x \ast (y \ast z) = (x \ast y) \ast z$
    \item Commutative: $x \ast y = y \ast x$
    \item Distributive: $x \ast (y + z) = x \ast y + x \ast z$
\end{itemize}

\pagebreak

\header{Fourier Series (Continuous Time)}

Given a correct periodic signal, one can describe it as a sum of complex exponentials,
called a \textbf{Fourier Series}. 
\gap
Signals that can be expressed as a fourier series must obey the \underline{Dirichlet Conditions}:
\gap
If over one period, a signal $x(t)$:
\begin{enumerate}
    \item Is single-valued
    \item Is absolutley integrable
    \item Has a finite number of maxima and minima
    \item Has a finite number of discontinuities
\end{enumerate}
The Dirichlet conditions are sufficient but \underline{not necessary}, however. This
being said, they can tell us that the Fourier series \underline{converges} to:
\begin{itemize}
    \item $x(t)$ where it is continuous
    \item half the value of the jump if it is discontinuous.
\end{itemize}


The Fourier Series has two parts to it:
\begin{multicols}{2}
    \centertext{\underline{Fourier Synthesis Equation}}
    \[
        x(t) = \sum_{k = -\infty}^\infty c_k e^{jk\omega t}    
    \]
    \centertext{where $\omega$ is the fundamental frequency of the function.}
    \vfill\null\columnbreak
    \centertext{\underline{Fourier Analysis Equation}}
    \[
        c_k = \frac{1}{T}\int_T e^{-jk\omega t}x(t){d}t    
    \]
    \centertext{where $T$ is the period of the function.}
    \vfill\null
\end{multicols}

\underline{Gibbs Phenomena:} If a function represented by a Fourier series has discontinuities,
there will be ``imperfections'' or ``ringings'' in the Fourier series. This is known
as the Gibbs Phenomena, where it is known that there will be ``spikes'' of about $9\%$
the height of the discontinuity around the bounds of the discontinuity itself.
\gap
\header{Operations on Fourier Series}

\underline{Summation of Signals}
\smallskip\\
Given two signals with the same period:
\begin{align*}
    x_1(t) = \sum_{k = -\infty}^\infty c_{k(1)}e^{jk\omega t} && x_2(t) = \sum_{k = -\infty}^\infty c_{k(2)}e^{jk\omega t}
\end{align*}
Then we can find their sum as:
\begin{align*}
    y(t) = Ax_1(t) + Bx_2(t) = \sum_{k = -\infty}^\infty c_k e^{jk\omega t} && c_k = Ac_{k(1)} + Bc_{k(2)}
\end{align*}
\underline{Time Shifting}
\smallskip\\
Given a signal, its timeshift, $x(t) \to x(t - t_0)$ can be obtained by:
\begin{align*}
    x(t - t_0) = \sum_{k = -\infty}^\infty c'_k e^{jk\omega t} && c'_k = e^{-jk\omega t_0}\cdot c_k
\end{align*}
\underline{Convolution of Signals}
Given two signals with the same fundamental frequency*:
\begin{align*}
    x_1(t) = \sum_{k = -\infty}^\infty c_{k(1)}e^{jk\omega t} && x_2(t) = \sum_{k = -\infty}^\infty c_{k(2)}e^{jk\omega t}
\end{align*}
Then the convolution of the signals takes the form of:
\begin{align*}
    y(t) = x_1(t) \ast x_2(t) = \sum_{k = -\infty}^\infty c_k' e^{jk\omega t} && c'_k = \sum_{l = -\infty}^\infty a_l b_{k - l}
\end{align*}
\gap
\header{Signal Power and Energy}

We can define the power and energy of a periodic function over a single period. 
\begin{multicols}{2}
    \centertext{\underline{Energy of a Signal}}
    \[
        E = \int_T \left|x(t)\right|^2 dt    
    \]
    \vfill\null\columnbreak
    \centertext{\underline{Power of a Signal}}
    \[
        P = \frac{1}{T}\int_T \left|x(t)\right|^2 dt    
    \]
    \vfill\null
\end{multicols}

Power and energy can also be used to show \underline{Parseval's Relation}.

\[
    \frac{1}{T}\int_T\left|x(t)\right|^2 dt = \sum_{k = -\infty}^\infty |c_k|^2    
\]

\header{Discrete Time Fourier Series}
\underline{Precursor -- Differences between DT and CT periodic signals}: There are 
two main differences between DT and CT periodic signals. 
\begin{enumerate}
    \item Frequency does not increase infinitley with $\omega$.
    \smallskip\\
    For some $x[n]$ with period T:
    \begin{align*}
        x[n] &= e^{\frac{2\pi n}{T}j}\\
        &= e^{\frac{2\pi (n + N)}{T}j}\\
        &= e^{\frac{2\pi n}{T}j}e^{\frac{2\pi N}{T}j}\\
        & \textrm{But if } N > T,\\
        &= e^{\frac{2\pi n}{T}j}e^{\frac{2\pi (T\alpha + N')}{T}j}\\
        &= e^{\frac{2\pi n}{T}j}e^{\frac{2\pi (N')}{T}j}e^{{2\pi \alpha}j}\\
        &= e^{\frac{2\pi n}{T}j}e^{{2\pi (N')}}
    \end{align*}
    It is important to note that this same result does not hold in CT because the
    resolution of $\alpha$ is not restricted to $\mathbb{Z}$.
    \item $\frac{\omega}{2\pi}$ must be rational for the signal to be periodic.
    \gap
    \underline{Note:} If $\frac{\omega}{2\pi}$ is a fraction, the numerator is the period.
\end{enumerate}
Now, we can define the Fourier synthesis and analysis equations for discrete time.
\begin{multicols}{2}
    \centertext{\underline{Fourier Synthesis Equation}}
    \[
        x[n] = \sum_{k = 0}^{N - 1}c_k e^{jk \frac{2\pi n}{N}}   
    \]
    \centertext{where $N$ is the period or number of samples of the function.}
    \vfill\null\columnbreak
    \centertext{\underline{Fourier Analysis Equation}}
    \[
        c_k = \frac{1}{N}\sum_{n = 0}^{N - 1}x[n]e^{-jk \frac{2\pi n}{N}}   
    \]
    \centertext{where $N$ is the period or number of samples of the function.}
    \vfill\null
\end{multicols}

\header{The Frequency Response}

The \textbf{Frequency Response} or the \textbf{System Response} of a system, denoted
by $H(j\omega)$ describes how frequencies are attenuated, amplified, and phase shifted
in a system. 
\smallskip\\
It is computed by:
\begin{align*}
    H(j\omega) = \int_{-\infty}^\infty e^{j\omega \tau}h(\tau){d}\tau 
\end{align*}
\underline{Question:} Are there frequency responses in DT?
\gap
\underline{Using the Frequency Response:} To use the frequency response, to compute
how a system changes a signal, there are three steps.
\begin{enumerate}
    \item Compute the frequency response (if it isn't already given).
    \item Compute the Fourier coefficients of the signal.
    \item Apply the frequency response to each of the Fourier coefficents to obtain
    the output signal.
\end{enumerate}
This can be generalized to:
\begin{align*}
    x(t) \to y(t) = \sum_{k = -\infty}^\infty c_k H(jk \omega)e^{jk\omega t} &&
    x[n] \to y[n] = \sum_{k = 0}^{N - 1}c_k H(e^{jk\omega})e^{jk\omega n}
\end{align*}

\header{Filters}
We can characterize filters by their frequency responses, $H$.
\gap
\underline{Filters in Discrete Time:} It is important to note that a filter in DT
is mirrored across zero. Frequencies increase up until $\omega = \pi$ and then
subsequently decrease.
\gap
\underline{Categories of Filters:} Filters can be broken up into two main categories:
\begin{enumerate}
    \item Infinite Impulse Response (IIR)\\
    Infinite impulse response filters are those whose impulse response \underline{does not}
    become zero over a finite amount of time.
    \item Finite Impulse Response (FIR)\\
    Finite impulse response filters are thos whose impulse response \underline{does} 
    become zero over a finite amount of time.
\end{enumerate}

\pagebreak

\header{The Fourier Transform}
The \textbf{Fourier Transform}, not to be confused with the \textit{Fourier series}
employs the principles of the \textit{Fourier series} to \underline{aperiodic signals}.
Thus, the equations governing it are:

\begin{multicols}{2}
    \centertext{\underline{Inverse Fourier Transform}}
    \[
        x(t) = \frac{1}{2\pi}\int_{-\infty}^\infty X(j\omega)e^{j\omega t}{d}\omega
    \]
    \vfill\null\columnbreak
    \centertext{\underline{Fourier Transform (Fourier Spectrum)}}
    \[
        X(\jomega) = \int_{-\infty}^\infty x(t)e^{-\jomega t}{d}t  
    \]
    \vfill\null
\end{multicols}
A more physical intuition of the Fourier transform is that it takes
a signal and decomposes it into a spectrum of its frequencies. This
is called the \underline{Fourier Spectrum}.
\gap
\underline{Fourier Transform and Impulse Response:} The Fourier
transform can be used to obtain the frequency response from the impulse response
and vice versa. This forwards direction of this was already known however.

\begin{multicols}{2}
    \centertext{\underline{Continuous Time}}
    \begin{align*}
        h[n] &= \frac{1}{2\pi}\int_{2\pi}H(e^\jomega)e^{\jomega n} \domega\\
        H(\jomega) &= \int_{-\infty}^\infty e^{-\jomega \tau}h(\tau)d\tau
    \end{align*}
    \vfill\null\columnbreak
    \centertext{\underline{Discrete Time}}
    \begin{align*}
        h(t) &= \frac{1}{2\pi}\int_{-\infty}^\infty e^{j\omega t}H(\jomega)\domega \\ 
        H(e^{\jomega}) &= \sum_{n = -\infty}^\infty e^{-\jomega n }h[n]
    \end{align*}
    \vfill\null
\end{multicols}

\end{document}