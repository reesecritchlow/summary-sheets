\documentclass{article}
\usepackage[utf8]{inputenc}
\usepackage[letterpaper, portrait, margin=1in]{geometry}
\usepackage{multicol}
\usepackage{amsmath}
\usepackage{amssymb}
\usepackage{enumerate}
\setlength\parindent{0pt}
\usepackage{enumerate}
\usepackage{graphicx}
\graphicspath{ {./images/} }
\usepackage{fancyhdr}
\usepackage{tcolorbox}
\hyphenchar\font=-1
\usepackage{tabularx}
\newcommand{\header}[1]{\begin{large}\noindent #1\end{large}\\\rule{\textwidth}{0.5pt}}
\newcommand{\gap}{\medskip\\}
\newcommand{\centertext}[1]{\begin{center}#1\end{center}}
\newcommand{\bfrac}[2]{\left(\frac{#1}{#2}\right)}
\newcommand{\formula}[3]{\begin{center} \begin{tcolorbox}[title = #2] $$#3$$\end{tcolorbox}\end{center}}
\newcommand{\where}{\hspace{0.5cm} \textrm{where} \hspace{0.5cm}}
\newcommand{\hgap}{\hspace{0.5cm}}
\newcommand{\pfrac}[2]{\frac{\partial #1}{\partial #2}}
\newcommand{\sheader}[1]{\underline{#1:}}
\newcommand{\doubleformula}[4]{\begin{center} \begin{tcolorbox}[title = #2] $$#3$$\\$$#4$$\end{tcolorbox}\end{center}}
\newcommand{\curly}[1]{\left\{#1\right\}}
\newcommand{\proj}[2]{{}\textrm{proj}_{#1}\left(#2\right)}
\newcommand{\subcell}[2]{\begin{tabular}{@{}c@{}}#1 \\ #2\end{tabular}}
\newcommand{\threecell}[3]{\begin{tabular}{@{}c@{}}#1 \\ #2 \\ #3\end{tabular}}
\newcommand{\fourcell}[4]{\begin{tabular}{@{}c@{}}#1 \\ #2 \\ #3 \\ #4\end{tabular}}

\usepackage{tikz}

\usepackage{scrextend}
\usepackage{lipsum}% for demo only!

\newcommand{\ds}{\displaystyle}
\newcommand{\Arg}{\textrm{Arg}}
\newcommand{\Log}{\textrm{Log}}

\newcommand{\tripleformula}[5]{\begin{center} \begin{tcolorbox}[title = #2] $$#3$$\\$$#4$$\\$$#5$$\end{tcolorbox}\end{center}}

\begin{document}
    \begin{center}
        \Large Math 305 Review Notes\\
        \normalsize Reese Critchlow
    \end{center}

    \header{Complex Numbers}

    At this point in the course, there we define only one representation of the 
    the imaginary unit, $i$:
    \begin{align*}
        i^2 = -1.
    \end{align*}
    For the remainder of this document, we will represent complex numbers $z$ as:
    \begin{align*}
        z = x + iy,
    \end{align*}
    where $i$ is the imaginary unit.
    \gap
    Hence, with $i$, we can define some other important properties:
    \begin{enumerate}
        \item $\ds \overline{\overline{z}} = z$
        \item $\ds \overline{z_1 \cdot z_2} = \overline{z_1}\cdot \overline{z_2}$ and
        $\ds \overline{\left(\frac{z_1}{z_2}\right)} = \frac{\overline{z_1}}{\overline{z_2}}$
        \item $\ds |z_1 \cdot z_2 | = |z_1| \cdot |z_2|$ and $\ds \left| \frac{z_1}{z_2} \right| = \frac{|z_1|}{|z_2|}$
    \end{enumerate}

    \sheader{Inequalities and Complex Numbers}
    Building off of the triangle inequality, we can define inequalities 
    for complex numbers:
    \begin{align*}
        &|z_1 + z_2| \leq |z_1| + |z_2| &|z_1 - z_2| &\geq \left||z_1| - |z_2|\right|\\
        &\textrm{(triangle inequality)} &|z_1 + z_2| &\geq \left| |z_1| - |z_2| \right|
    \end{align*}
    As a result, we can bound the modulus of the sum of two complex numbers as:
    \begin{align*}
        \left| |z_1| - |z_2| \right| \leq |z_1 + z_2| \leq |z_1| + |z_2|.
    \end{align*}
    If trying to obtain a bound for the sum of multiple complex numbers, it is important
    to always obtain the maximal/minimal bounds for each case when aggregating 
    complex numbers.
    \gap
    \sheader{Representations of Planar Sets in Complex Numbers} Taking the 
    prior definition of $z = x + iy$, and interpreting the $x$ value as an $x$ coordinate,
    and the same for $y$, then we can define planar sets in terms of complex numbers.
    To start off, we first define how to obtain the $x$ and $y$ values of a complex number:
    \begin{align*}
        \textrm{Re}(z) &= x = \frac{z + \overline{z}}{2} & \textrm{Im}(z) &= x = \frac{z - \overline{z}}{2i}
    \end{align*}
    With this definition, we can define some common representations:
    \begin{enumerate}
        \item \sheader{Circles in $\mathbb{R}^2$} 
        \begin{align*}
            (x - x_0)^2 + (y - y_0)^2 = r_0^2 &\iff |z - z_0| = r_0
        \end{align*}
        \item \sheader{Lines in $\mathbb{R}^2$}
        \begin{align*}
            ax + by = c \iff a \frac{z + \overline{z}}{2} + b \frac{z - \overline{z}}{2i} = c
        \end{align*}
        \item \sheader{Ellipses in $\mathbb{R}^2$}
        \begin{align*}
            \frac{x^2}{a^2} + \frac{y^2}{b^2} = 1 \iff |z - F| + |z + F| = 2a 
        \end{align*}
        Where $F = \sqrt{a^2 - b^2}$. 
        \gap
        It is to be noted that this representation only allows for horizontal shifts.
        One can get vertical shifts by using an alternate form:
        \begin{align*}
            \frac{x^2}{a^2} + \frac{y^2}{b^2} = 1 \iff |z - Fi| + |z + Fi| = 2b.
        \end{align*}
        Shifting can be observed when some representation $|z - F_1| + |z + F_2| = 2a$,
        where $F_1 \neq F_2$. The shift can be obtained by averaging $F_1$ and $F_2$. 
    \end{enumerate}
    A common example of a coordinate transform is to square both sides and convert 
    into a circle. This is because we require $r_0$ to be squared.
    \gap
    
    \header{Polar Coordinates and Arguments}

    One can express a complex number $z$ in polar coordinates using Euler's formula:
    \begin{align*}
        e^{i \phi} = \cos \phi + i \sin \phi.
    \end{align*}
    Hence, it is implied that a representation of a complex number can be as follows:
    \begin{align*}
        z = re^{i \phi}.
    \end{align*}
    Where $r$ is known as the \underline{modulus} of $z$ and $\phi$ is known as the 
    \underline{modulus} of $z$. 
    \begin{align*}
        r &= \sqrt{x^2 + y^2} & \phi &= \arg(z)
    \end{align*}

    \sheader{Arguments} There exist two different types of arguments:
    \begin{enumerate}
        \item \sheader{Principal Argument} $\textrm{Arg}(z) = \phi \in (-\pi, \pi]$. \\
        The principal argument is single-valued and unique.
        \item \sheader{General Argument*} $\arg(z) = \textrm{Arg}(z) \pm 2\pi k, k \in \mathbb{Z}$.\\
        The general argument can attain infinite different values.  
    \end{enumerate}
    \sheader{Finding the Principal Argument} The Principal Argument, as a general 
    form, is given by: 
    \begin{align*}
        \textrm{Arg}(z) = \arctan\left(\frac{y}{x}\right) + m\pi, m \in \curly{-1, 0, 1}.
    \end{align*}
    Hence, we define cases for the correct value of $m$, based on the location of $z$.
    \begin{enumerate}
        \item \sheader{Quadrant I} $m = 0$, $\textrm{Arg}(z) = \arctan\left(\frac{y}{x}\right)$
        \item \sheader{Quadrant II} $m = 1$, $\textrm{Arg}(z) = \arctan\left(\frac{y}{x}\right) + \pi$
        \item \sheader{Quadrant III} $m = -1$, $\textrm{Arg}(z) = \arctan\left(\frac{y}{x}\right) - \pi$.
        \item \sheader{Quadrant IV} $m = 0$, $\textrm{Arg}(z) = \arctan\left(\frac{y}{x}\right)$
    \end{enumerate}
    It is important to note that the argument of zero is undefined.
    \gap
    \sheader{Properties of Complex Numbers in Polar Forms}
    \begin{enumerate}
        \item $e^{2k\pi i} = 1$
        \item $e^{i\phi_1} e^{i \phi_2} = e^{i (\phi_1 + \phi_2)}$
        \item $\overline{e^{i\phi}} = e^{-i \phi}$
        \item $|e^{i \phi}| = 1$
    \end{enumerate}
    
    \sheader{Properties of Arguments}
    \begin{enumerate}
        \item $\Arg(z_1 \cdot z_2) \neq \Arg(z_1) + \Arg(z_2)$
        \item $\Arg(z_1 \cdot z_2) = \Arg(z_1) + \Arg(z_2) + 2k\pi, k \in \mathbb{Z}$
        \item $\arg(z_1 \cdot z_2) = \arg(z_1) + \arg(z_2)$
        \gap
        It is important to note that (3) is \textit{set equality}, not value equality.
    \end{enumerate}

    \header{Powers of Complex Numbers}
    Certain properties arise from taking powers of complex numbers.
    \gap
    \sheader{De Moirre's Formula}
    \begin{align*}
        (\cos \phi + i \sin \phi)^N = \cos(N \phi) + i\sin(N \phi)
    \end{align*}

    This can often be combined with the \underline{binomial theorem} to derive 
    trig identities.
    \gap
    \sheader{Binomial Theorem}
    \begin{align*}
        (x + y)^n = \sum_{k = 0}^n \binom{n}{k} x^k y^{n-k}
    \end{align*}
    Where $\ds \binom{n}{k} = \frac{n!}{k!(n-k)!}$.

    \sheader{Roots of $z$}
    We can define the roots of $z$, $z^{\frac{1}{n}}$ as:
    \begin{align*}
        z_0^{\frac{1}{n}} &= r_0^{\frac{1}{n}}e^{i\left(\frac{\phi_0}{n} + \frac{2k\pi}{n}\right)} & k = 0, 1, \ldots , n-1
    \end{align*}
    We also define the \underline{principle value} of a root to be the one corresponding 
    to $k = 0$.
    \gap
    \sheader{Raising Complex Numbers as Powers} There are also interesting consequences
    of raising complex numbers as powers. Take a complex number $z = x + iy$:
    \begin{align*}
        e^z &= e^{x + iy}\\
        e^z &= e^{x} \cdot e^{iy}\\
        e^z &= e^x(\cos y + i\sin y).
    \end{align*}
    Thus, we also obtain some properties:
    \begin{enumerate}
        \item $\ds e^{z_1 + z_2} = e^{z_1}\cdot e^{z_2}$
        \item $\ds e^{z_1 - z_2} = \frac{e^{z_1}}{e^{z_2}}$
        \item $\ds e^{\overline{z}} = \overline{e^z}$
        \item $|e^z| = e^x$
    \end{enumerate}

    \header{Functions of Complex Numbers}

    We can describe functions of complex numbers as:
    \begin{align*}
        w = f(z),
    \end{align*}
    where $w = u + iv$ and $z = x + iy$. Different functions have different images.
    To determine the image of a given function, we can use the following approach:
    \begin{enumerate}
        \item Solve for $z$ in terms of $w$ from $f(z) = w$.
        \item Substitute each $z$ for the $z(w)$ expression in the set notation.
    \end{enumerate}
    \pagebreak
    Different Transforms:
    \begin{enumerate}
        \item $w = Az + B$ transforms circles to circles, lines to lines.
        \item $w = \frac{1}{z}$ transforms \{circles or lines\} to \{circles or lines\}.
        \item Mobius Transform, $w = \frac{az + b}{cz + d}$ transforms \{circles or lines\} to \{circles or lines\}.
        \item $w = z^n$ Power Transforms generally rotate sets.
        \item $w = e^z$ Exponential Transofrms generally turn lines into circles and vice versa.
    \end{enumerate}

    \header{Derivatives of Complex Functions}
    Like single-variable calculus, there are some basic definitions that need to 
    be highlighted in complex-variable calculus.
    \gap
    \sheader{Limits}
    We can define the limit of a complex function as:
    \begin{align*}
        \lim_{z \to z_0} f(z) = \lim_{(x, y) \to (x_0, y_0)}u(x, y) + i \lim_{(x, y) \to (x_0, y_0)}v(x, y)
    \end{align*}
    However, an additional layer of complexity is introduced here. $(x, y) \to (x_0, y_0)$
    implies that any path can be taken to get to $(x,y)$. Thus, strange behaviours occur.
    \gap
    As an aside, we can note that usual properties of limits generally still hold 
    with complex functions:
    \begin{enumerate}
        \item $\ds \lim \frac{f(z)}{g(z)} = \frac{\lim f(z)}{\lim g(z)}$
        \item $\lim(c_1 f + c_2 g) = c_1 \lim f(z) + c_2 \lim g(z)$.
    \end{enumerate}
    \underline{Continuity of Functions}
    \begin{align*}
        f(z) \textrm{ is continuous at }z_0 \iff \lim_{z \to z_0} f(z) = f(z_0) \iff \left(\lim_{z \to z_0} u(z_0) = u(z_0) \wedge \lim_{z \to z_0} v(z_0) = v(z_0)\right)
    \end{align*}
    It is important to note that the choice of path is important here. 
    If two different paths lead to different values in the limit, the limit does not 
    exist.
    
    \sheader{Differentiability}
    \begin{align*}
        \lim_{z \to z_0} \frac{f(z) - f(z_0)}{z - z_0} \iff \lim_{\Delta z \to 0} \frac{f(z_0 + \Delta z) - f(z_0)}{\Delta z} \iff f(z) \textrm{ is differentiable.}
    \end{align*}
    Where the inclusion of a limit means that the limits must exist.
    \gap
    \sheader{Cauchy-Riemann Equation}
    A helpful tool for governing differentiability is the Cauchy-Riemann Equation:
    \begin{align*}
        \begin{cases}
            \frac{\partial u}{\partial x} = \frac{\partial v}{\partial y}\\
            \frac{\partial u}{\partial y} = -\frac{\partial v}{\partial x}
        \end{cases}
    \end{align*}
    Implications of the CRE:
    \begin{itemize}
        \item $(\implies)$: If some $f(z)$ is differentiable at some $z_0$, then 
        the CRE is satisifed.
        \item $(\impliedby)$: If some function $f(z)$ satisfies the CRE at some $z_0$,
        AND the derivative is continuous, then $f(z)$ is differentiable.
    \end{itemize}

    There are some additional definitions here too:
    \begin{enumerate}
        \item A function $f$ is \underline{analytic} at a point $z_0$ \underline{if} $f$ is differentiable in a neigbourhood of $z_0$.
        \item A function $f$ is \underline{entire} if it is analytic everywhere.
    \end{enumerate}
    Being differentiable at a single point is not analytic.
    \gap
    Taking the derivative of complex functions is the same as in 
    non-complex functions. The product, quotient, and chain rules all hold.
    \pagebreak
    \sheader{Remark}
    \begin{align*}
        f \textrm{ is differentiable} \implies \frac{\partial f}{\partial \overline{z}} = 0 \textrm{ at } z_0
    \end{align*}
    \sheader{Consequences of the Cauchy-Riemann Equation}
    \begin{enumerate}
        \item The CRE guarantees the satisfaction of the laplace equation if a 
        function is entire. Thus, we can say that $v$ is the \underline{harmonic conjugate}
        of $u$, and vice versa.
    \end{enumerate}

    \header{Functions and Mappings}

    We can define a variety of different functions which map various domains to 
    various images. Expanding off of what was already stated in the ``Functions of
    Complex Numbers'' section, we can state some more functions:
    \gap
    \bgroup
    \def\arraystretch{1.5}% 
    \begin{tabular}{ |l|l|l|l| } 
        \hline
        \underline{Name} & \underline{Function} & \underline{Mapping} & \underline{Expanded Form} \\
        \hline
        Square  & $w = z^2$  & Scales modulus and argument & $w = \begin{cases}
            u = x^2 - y^2\\
            v = 2xy
        \end{cases}$   \\
        \hline
        Exponential & $w = e^z$ & Transforms planes/circles into planes/circles & $w = \begin{cases}
            u = e^x \cos y\\
            v = e^x \sin y\\
        \end{cases}$  \\
        \hline
    \end{tabular}
    \egroup
    \gap
    
    \header{Inverse Functions}
    \bgroup
    \def\arraystretch{1.5}% 
    \begin{tabular}{ |c|c| } 
        \hline
        \underline{Function} & \underline{Inverse} \\
        \hline
        $ \sin(z)$ & $\sin^{-1}z = -i\cdot\Log(iz + (1-z^2)^{\frac{1}{2}})$ \\
        & where $(1 - z^2)^{\frac{1}{2}} = i|z-1|^\frac{1}{2} |z + 1|^\frac{1}{2}e^{i\left(\frac{\phi_1 + \phi_2}{2}\right)} $\\
        & and $2\pi < \phi_1 < 4\pi$\\
        & $ -\pi < \phi_2 < \pi$\\
        \hline
        $w = e^z$ & $z = \Log(w)$\\
        \hline
        $w = z^\alpha$ & $e^{\frac{1}{\alpha} \Log z} = z^{\frac{1}{\alpha}}$\\
        \hline
    \end{tabular}
    \egroup

    \pagebreak

    \centertext{\Large Course Review}
    \vspace{0.5cm}
    \header{Elementary Functions and Mapping Properties}
    \vspace{0.1cm}
    \bgroup
    \def\arraystretch{1.5}%

    \begin{tabular}{ |l|l|p{5.5cm}|l| }
        \hline
        \underline{Name} & \underline{Function} & \underline{Description of Transform} & \underline{Expanded Form} \\
        \hline
        %% how can i make two rows in a single cell?
        Linear & $w = az + b$ &  \subcell{Lines $\to$ Lines}{Circles $\to$ Circles} & $w = \begin{cases}
            u = ax + b\\
            v = ay + b
        \end{cases}$ \\
        \hline
        Mobius Transform & $w = \frac{az + b}{cz + d}$ & Lines/Circles $\to$ Lines/Circles & $w = \begin{cases}
            u = \frac{ax + b}{cx + d}\\
            v = \frac{ay + b}{cy + d}
        \end{cases}$ \\
        \hline
        Exponential & $w = e^z$ & \subcell{Lines $\to$ Cirlces}{Circles $\to$ Lines} & $w = \begin{cases}
            u = e^x \cos y\\
            v = e^x \sin y\\
        \end{cases}$ \\
        \hline
        Joukowsky Map & $w = \frac{1}{2}\left(z + \frac{1}{z}\right)$ & Transforms a semi-circle of radius 1, centered at the origin to a flat line along the $x$-axis & 
        \\\hline
        Sine & $w = \sin z$ & Transforms an infinite 3-sided box with sides at $-\pi/2, \pi/2$ centered at $0$ into a line on the $x$-axis. & $w = \begin{cases}
            u = \sin x \cosh y\\
            v = \cos x \sinh y
        \end{cases}$ \\
        \hline
        Cosine & $w = \cos z$ & I forgor & $w = \begin{cases}
            u = \cos x \cosh y\\
            v = -\sin x \sinh y
        \end{cases}$ \\
        \hline
        Phase Rotation & $w = z \cdot e^{i\phi}$ & Rotates the plane by $\phi$ radians. & $w = \begin{cases}
            u = x \cos \phi - y \sin \phi\\
            v = x \sin \phi + y \cos \phi
        \end{cases}$ \\
        \hline
        Polynomial & $w = z^n$ & Scales angles and magnitudes by $n$. & $w = \begin{cases}
            u = x^n\\
            v = y^n
        \end{cases}$ \\
        \hline
    \end{tabular}
    \egroup
    \vspace{0.5cm}

    \header{Inverse Functions and Branch Cuts}
    \vspace{1pt}

    \bgroup
    \def\arraystretch{1.5}%
    \begin{tabular}{ |l|l|l| }
        \hline
        \underline{Multi-Valued Function} & \underline{Single-Valued Function} & \underline{Inverse of:}\\
        \hline
        $\text{arg}(z)$ & $\text{Arg}(z), -\pi < \phi < \pi$ & -- \\
        \hline 
        $\text{log}(z) = \ln(|z|) + i(\text{Arg}(z) + 2\pi k)$ & $\text{Log}(z) = \ln(|z|) + i\cdot \text{Arg}(z)$ & $e^w = z$ \\ 
        \hline
        $\sin^{-1}(z)$ & \fourcell{$\sin^{-1}(z)= -i\cdot\Log(iz + i(z^2 - 1)^{\frac{1}{2}})$}{$(z^2 - 1)^{\frac{1}{2}} = |z-1|^\frac{1}{2}|z+1|^\frac{1}{2}e^{i\frac{\phi_1 + \phi_2}{2}}$}{$2\pi < \phi_1 < 4\pi$}{$ -\pi < \phi_2 < \pi$} & $\sin(w) = z$ \\
        \hline
        $\cos^{-1}(z)$ & \fourcell{$\cos^{-1}(z) = -i\cdot\Log(z + (z^2 - 1)^{\frac{1}{2}})$}{$(z^2 - 1)^{\frac{1}{2}} = |z-1|^\frac{1}{2}|z+1|^\frac{1}{2}e^{i\frac{\phi_1 + \phi_2}{2}}$}{$0 < \phi_1 < 2\pi$}{$ -\pi < \phi_2 < \pi$} & $\cos(w) = z$ \\
        \hline
    \end{tabular}

    \pagebreak

    \header{Inequality Reminders}
    \begin{itemize}
        \item $|e^z| \leq e^{\text{Re}(z)}$
        \item $|z^n + \cdots | \leq |z|^n + \cdots $
        \item $|z^n + \cdots | \geq |z|^n - \cdots $
    \end{itemize}
    
    \header{Differentiability, Analyticity, Cauchy-Riemann Equations}
    Like single-variable calculus, there are some basic definitions that need to 
    be highlighted in complex-variable calculus.
    \gap
    \sheader{Limits}
    We can define the limit of a complex function as:
    \begin{align*}
        \lim_{z \to z_0} f(z) = \lim_{(x, y) \to (x_0, y_0)}u(x, y) + i \lim_{(x, y) \to (x_0, y_0)}v(x, y)
    \end{align*}
    However, an additional layer of complexity is introduced here. $(x, y) \to (x_0, y_0)$
    implies that any path can be taken to get to $(x,y)$. Thus, strange behaviours occur.
    \gap
    As an aside, we can note that usual properties of limits generally still hold 
    with complex functions:
    \begin{enumerate}
        \item $\ds \lim \frac{f(z)}{g(z)} = \frac{\lim f(z)}{\lim g(z)}$
        \item $\lim(c_1 f + c_2 g) = c_1 \lim f(z) + c_2 \lim g(z)$.
    \end{enumerate}
    \underline{Continuity of Functions}
    \begin{align*}
        f(z) \textrm{ is continuous at }z_0 \iff \lim_{z \to z_0} f(z) = f(z_0) \iff \left(\lim_{z \to z_0} u(z_0) = u(z_0) \wedge \lim_{z \to z_0} v(z_0) = v(z_0)\right)
    \end{align*}
    It is important to note that the choice of path is important here. 
    If two different paths lead to different values in the limit, the limit does not 
    exist.
    
    \sheader{Differentiability}
    \begin{align*}
        \lim_{z \to z_0} \frac{f(z) - f(z_0)}{z - z_0} \iff \lim_{\Delta z \to 0} \frac{f(z_0 + \Delta z) - f(z_0)}{\Delta z} \iff f(z) \textrm{ is differentiable.}
    \end{align*}
    Where the inclusion of a limit means that the limits must exist.
    \gap
    \sheader{Cauchy-Riemann Equation}
    A helpful tool for governing differentiability is the Cauchy-Riemann Equation:
    \begin{align*}
        \begin{cases}
            \frac{\partial u}{\partial x} = \frac{\partial v}{\partial y}\\
            \frac{\partial u}{\partial y} = -\frac{\partial v}{\partial x}
        \end{cases}
    \end{align*}
    Implications of the CRE:
    \begin{itemize}
        \item $(\implies)$: If some $f(z)$ is differentiable at some $z_0$, then 
        the CRE is satisifed.
        \item $(\impliedby)$: If some function $f(z)$ satisfies the CRE at some $z_0$,
        AND the derivative is continuous, then $f(z)$ is differentiable.
    \end{itemize}

    There are some additional definitions here too:
    \begin{enumerate}
        \item A function $f$ is \underline{analytic} at a point $z_0$ \underline{if} $f$ is differentiable in a neigbourhood of $z_0$.
        \item A function $f$ is \underline{entire} if it is analytic everywhere.
    \end{enumerate}
    Being differentiable at a single point is not analytic.
    \gap
    Taking the derivative of complex functions is the same as in 
    non-complex functions. The product, quotient, and chain rules all hold.
    \\
    \sheader{Remark}
    \begin{align*}
        f \textrm{ is differentiable} \implies \frac{\partial f}{\partial \overline{z}} = 0 \textrm{ at } z_0
    \end{align*}
    \sheader{Consequences of the Cauchy-Riemann Equation}
    \begin{enumerate}
        \item The CRE guarantees the satisfaction of the laplace equation if a 
        function is entire. Thus, we can say that $v$ is the \underline{harmonic conjugate}
        of $u$, and vice versa.
        \item A function is said to be \underline{harmonic} if the laplace equation is 
        satisfied, that is if:
        \begin{align*}
            \Delta u = u_{xx} + u_{yy} = 0 \hspace{0.5cm} \textrm{ or } \hspace{0.5cm} \Delta v = v_{xx} + v_{yy} =  0
        \end{align*}
    \end{enumerate}

    \header{First Steps into Integrals: Path Integrals and the Fundamental Theorem of Calculs}
    \sheader{Path Integrals}
    We can take the \underline{path integral} of a complex function $f(z)$ over some 
    path $C$:
    \begin{align*}
        \int_C f(z) dz = \int_{t_1}^{t_2} f(z(t)) \cdot z'(t)dt
    \end{align*}
    Where $f(t)$ is some parametrization of the path $C$.
    \gap
    \sheader{Fundamental Theorem of Calculus}
    Like in real calculus, we can use antiderivatives to integrate a function.
    \gap
    Given if an antiderivative $F(z)$ exists for some function $f(z)$, then 
    \begin{align*}
        \int_C f(z) dz = F(z_f) - F(z_i)
    \end{align*}
    It is to be noted that in order for the antiderivative to exist, 
    $f(z)$ must be analytic on $C$.
    \gap
    \header{The Cauchy Integral Formula}
    \sheader{Cauchy's Theorem} Before the introduction of the Cauchy Integral Formula,
    we first introduce \underline{Cauchy's Theorem}. It is as follows:
    \gap
    Let a path $C$ be a simple closed loop. Suppose a function $f(z)$ is analytic 
    on and inside $C$. Then:
    \begin{align*}
        \int_C f(z)dz = 0
    \end{align*}
    Consequently, we can also obtain the fact that for some function $f(z)$ with a 
    discontinuity at $z_0$, then it follows that:
    \begin{align*}
        \int_{|z - z_0| = r} f(z) dz = 2\pi i
    \end{align*}

    \sheader{Deformation of Path} If there are two paths $C_0$ and $C_1$ on/in some 
    domain $C$, if $C$ is analytic and $C_0$ is contained in $C_1$ or vice versa, 
    then:
    \begin{align*}
        \int_{C_0}f(z)dz = \int_{C_1}f(z)dz
    \end{align*}
    Expanding on this, given some domain $C$, with $N$ ``holes'' in it $C_1, \ldots, C_N$,
    then the integral over $C$ can be expressed as:
    \begin{align*}
        \int_C f(z)dz = \sum_{j = 1}^N \int_{C_j}f(z)dz
    \end{align*}

    \pagebreak 

    \sheader{Cauchy Integral Formula} Finally, we arrive at the Cauchy Integral Formula.
    It is as follows:
    \begin{align*}
        f^{(m)}(z_0) = \frac{m!}{2\pi i} \int_C\frac{f(z)}{(z-z_0)^{m+1}}dz
    \end{align*}
    Consequently, we can rearrange this to be more applicable for integrals.
    \begin{align*}
        \int_C \frac{f(z)}{(z-z_0)^{m + 1}} = 2\pi i  \cdot \frac{f^{(m)}(z_0)}{m!}
    \end{align*}
    Finally, we can generalize:
    \begin{align*}
        \int_C f(z)dz = 2\pi i \sum_{j = 1}^N \frac{f^{(m_j)}_j(z_0)}{m_j!}
    \end{align*}
    Where each $f_j(z)$ is the original function $f(z)$ with the discontinuity component 
    $\frac{1}{z-z_k}$ divided out.
    \gap
    It is important to note that the Cauchy Integral Formula is only really good for 
    functions of the form:
    \begin{align*}
        f(z) = \frac{g(z)}{(z-z_0)^{m_0} \cdot (z-z_1)^{m_1} \cdot \cdots}
    \end{align*}

    \header{Applications and Consequences of the Cauchy Integral Formula}

    \sheader{Pointwise Estimate} For some function $f(z)$ on a circular domain $C: |z - z_0| = R$, then we can 
    establish an upper bound for the function and its derivatives evaluated at $z = z_0$:
    \begin{align*}
        f^{(m)}(z_0) \leq \frac{m!}{R^m}\max_{|z-z_0| = R}|f(z)|
    \end{align*}

    \sheader{Liouville Theorem} Given some function $f(z)$ that is entire and bounded, then 
    it is a constant function.
    \begin{align*}
        |f(z)| \leq M \implies f(z)\equiv C
    \end{align*}
    There exist some important variants of the Liouville Theorem:
    \begin{enumerate}
        \item $f = u + iv \wedge (u \geq -M \vee u \leq K) \implies f \equiv C$ \\
        Proof Mechanism: $g(z) = e^{-f(z)}$
        \item $f = u + iv \wedge au + bv \geq 0 \implies f \equiv D$\\
        Proof Mechanism: $g = e^{\alpha f}, |g| = |e^{\alpha f}| = e^{-(au + bv)}$
        \item $|f(z)| \leq C\cdot (1 + |z|)^n \implies $ f(z) must be a polynomial of degree lesser than $n$.
    \end{enumerate}

    \sheader{Maximum Principle} For some function $f(z)$ which is analytic on some domain 
    and its boundary $D \cup \partial D$, its maximum value is found on its boundary:
    \begin{align*}
        \max_{\overline{D}} |f(z)| = \max_{\partial D} |f(z)|.
    \end{align*}

    \sheader{Minimum Principle} For some function $f(z)$ which is analytic on some domain
    and its boundary $D \cup \partial D$, its minimum value is either zero, or it is 
    found on the boundary of the domain $\partial D$:
    \begin{align*}
        \left(\min_{\overline{D}} |f(z)| = 0 \right) \vee \left(\min_{\overline{D}}|f(z)| = \min_{\partial D} |f(z)|\right).
    \end{align*}
    
    \sheader{Argument Principle}
    For some function $f(z)$ which is analytic both on and inside $C$, with 
    a finite number of roots inside $C$ with multiplicity $n_j$, then:
    \begin{align*}
        \int_C \frac{f'(z)}{f(z)} = 2\pi i \cdot \sum_{j = 1}^k n_j
    \end{align*}
    Conequently, the argument principle follows that:
    \begin{align*}
        \frac{1}{2\pi i} \int_C \frac{f'(z)}{f(z)} = N = \frac{arg(f)|_{f(C)}}{2\pi}
    \end{align*}
    However, this theorem is not very applicable, but its consequenting theorems are.
    \gap
    \sheader{Rouche's Theorem (Application of Argument Principle)} 
    Given some function $f(z) = g(z) + h(z)$, if one of the terms inside $f(z)$
    dominates the others, that is $|f(z)-g(z)| < |g(z)|$ on $C$, then 
    the number of zeroes on $C$ for $f(z)$ is the same as that of the number of 
    zeroes on $C$ for $g(z)$, $N_f = N_g$.
    \gap
    \sheader{Nyquist Criterion (Application of Argument Principle)} 
    Given some polynomial function $p(z)$ with degree $n$, the number of zeroes on the right half plane $\curly{Re(z) > 0}$
    is equal to:
    \begin{align*}
        N = \frac{1}{2\pi}\left(n\pi + 2\cdot \text{arg}(p(z))|_{\Gamma_1}\right)
    \end{align*}
    Where $\Gamma_1 = \curly{z = iy, 0 < y < +\infty}$.
    \gap
    \header{Taylor and Laurent Series}

    In real calculus, the \underline{Taylor Series} was introduced:
    \begin{align*}
        f(z) = a_0 + a_1(z - z_0) + \cdots + a_n(z-z_0)^n \where a_n = \frac{f^{(n)}(z_0)}{n!} =
        \frac{1}{2\pi i}\int_{|z-z_0 = r}\frac{f(\eta)}{(\eta - z_0)^{n + 1}}d\eta
    \end{align*}
    We can state that the Taylor Series held for a function $f(z)$ which was 
    analytic on the domain $|z-z_0| < r_2$.
    \gap
    \sheader{Laurent Series} Complex analysis introduces the \underline{Laurent Series}, which
    serves two purposes:
    \begin{enumerate}
        \item Computes the negative power components of a series expansion of a function (e.g. $z^-n$)
        \item Computes an expansion for an annular domain: $C: r_1 < |z-z_0| < r_2$.
    \end{enumerate}
    Hence, the laurent series is as follows:
    \begin{align*}
        f(z) =&\,\, a_0 + a_1(z-z_0) + \cdots + a_n(z - z_0)^n + \cdots \\
        & + a_{-1}(z - z_0)^{-1} + \cdots + a_{-n}(z - z_0)^{-n}
    \end{align*}
    Where $\displaystyle a_n = \frac{1}{2\pi i} \int_{|z-z_0| = r} \frac{f(\eta)}{(\eta - z_0)^{n + 1}}d\eta$.
    \gap
    We can use the Laurent Series to also classify singularities: for a singularity $z = z_0$
    for some function $f(z)$, it follows that:
    \begin{enumerate}
        \item If there are no negative terms in the series $a_{-1} = \cdots = a_{-n} = 0$,
        then we classify $z_0$ as a \underline{removable singularity}.
        \item If the negative terms in the series terminate after some $a_{-m}$, then 
        we classify $z_0$ as a \underline{singularity of order $m$}.
        \item If the negative terms never terminate, then $z_0$ is classified as an \underline{essential singularity}.
    \end{enumerate}

    \pagebreak

    \header{Residues and the Cauchy Residue Theorem}

    \sheader{Residue} In the Laurent series, we call the $a_{-1}$ term the \underline{resiude}
    of a function $f(z)$ and a singularity $z_0$. We can compute it in a multitude of ways:

    \begin{addmargin}[2.5em]{0em}
        \sheader{Removable Singularities} $\ds \text{Res}(f(z), z_0) = \frac{P(z_0)}{Q'(z_0)}$
        \gap
        \sheader{Poles with Order $\geq 2$} There are two options:
        \begin{enumerate}
            \item $\ds \text{Res}(f(z), z_0) = \frac{1}{(m-1)!}f_0^{(m - 1)}(z_0)$ (like prior)
            \item Laurent Series
        \end{enumerate}
        \sheader{Essential Singularities} Laurent Series
    \end{addmargin}
    \vspace{0.5cm}
    \sheader{Determining Pole Order}
    \begin{enumerate}
        \item If the pole is in an ``easy'' form $(z-z_j)^{m_j}$ in denominator, 
        then the order of the pole is simply $m_j$.
        \item If not, then there are two options:
        \begin{enumerate}
            \item Take the Laurent Series and find the order of the largest negative exponent.
            \item (Hacky) Take successive derivatives of the denominator. The first 
            $n$-th derivative which does not evaluate to zero is the order.
        \end{enumerate}
    \end{enumerate}

    \sheader{Cauchy Residue Theorem}
    \begin{align*}
        \int_C f(z)dz = 2\pi i \sum_{j = 1}^N \text{Res}(f(z), z_j)
    \end{align*}
    \gap
    \header{Real Integrals}

    There exist some real integrals which are easily solvable using complex analysis.
    For these integrals, a simple problem solving framework exists:
    \begin{enumerate}
        \item Choose a contour that will make computation simple, based on the 
        form of the integrand.
        \item Write out the integral over each segment of the contour. Terms that 
        evaluate to the original integral, $I$, will appear.
        \item Set this equal to the Cauchy Residue theorem for the domain that the 
        contour encircles, and solve for $I$.
    \end{enumerate}

    \header{Fourier and Laplace Transforms} 
    (under construction)

    % \pagebreak


    % \sheader{Contours}
    % \bgroup
    % \def\arraystretch{3}

    % \begin{tabular}{|c|c|c|c|}
    %     \hline
    %     \underline{Integral Form} & \underline{Contour Name} & \underline{Function} & \underline{Solution}\\
    %     \hline
    %     $\displaystyle \int_0^{2\pi} F(\sin(\theta), \cos(\theta))d\theta$ & Full Circle & & $I = 2\pi i \sum_{j = 1}^N \text{Res}(f(z), z_j)$\\
    %     \hline
    %     \subcell{$\displaystyle \int_{-\infty}^{\infty}\frac{P(x)}{Q(x)}dx$}{$\deg(Q) \geq \deg(P) + 2$} & Upper Semi-Circle & & $ I = 2\pi i \sum_{j = 1}^N \text{Res}(f(z), z_j)$\\
    %     \hline
    %     \subcell{$\displaystyle \int_{-\infty}^{\infty} e^{i\beta x} \frac{P(x)}{Q(x)}dx$}{$\deg(Q) \geq \deg(P) + 1, \beta > 0 $} & & Upper Semi Circle & $ I = 2\pi i \sum_{j = 1}^N \text{Res}(f(z), z_j)$\\
    %     \hline
    %     \subcell{$\displaystyle \int_{-\infty}^\infty e^{i\beta x} \frac{P(x)}{Q(x)}dx$}{$\deg(Q) \geq \deg(P) + 1, \beta < 0 $} & Lower Semi Circle & & $ I = -2\pi i \sum_{j = 1}^N \text{Res}(f(z), z_j)$\\
    %     \hline
    %     $\int_{-\infty}^\infty F(e^x)dx $ & Box Contour &  \\

    % \end{tabular}




    \egroup



\end{document}