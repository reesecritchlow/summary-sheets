\documentclass{article}
\usepackage[utf8]{inputenc}
\usepackage[letterpaper, portrait, margin=0.5in]{geometry}
\usepackage{multicol}

\begin{document}

\noindent \large \textbf{Electric Force}\normalsize \qquad
Units: N\\
\noindent\rule{\textwidth}{0.5pt}

\begin{center}
\begin{multicols}{3}

Coulombs Law\\*
\footnotesize (Two Point Charges) \normalsize
\[
    \vec{F} = \frac{1}{4\pi\epsilon_0}\cdot\frac{|q_1q_2|}{r^2} = k\frac{|q_1q_2|}{r^2}
\]

Force Due to Electric Field
\[
    \vec{F} = q_0\vec{E}
\]

Force in a Uniform Field
\[
    F(z) = -\frac{dU}{dz}
\]
\end{multicols}
\end{center}
\medskip
\large \textbf{Electric Fields}\normalsize \qquad
Units: $\frac{\textrm{N}}{\textrm{C}}$ or $\frac{\textrm{V}}{\textrm{M}}$\\
\noindent\rule{\textwidth}{0.5pt}

\begin{center}
    \begin{multicols}{4}
        
        Electric Field From a Point Charge
        \[
            E = \frac{kq}{r^2}
        \]
        
        Electric Field From Force
        \[
            \vec{E} = \frac{\vec{F}_0}{q_0}
        \]
        
        Gauss's Law
        \[
            E = \frac{Q_{\textrm{encl}}}{\epsilon_0A}
        \]
        
        From Equipotential Lines
        \[
            E = -\nabla V
        \]
        
    \end{multicols}
    
    \begin{multicols}{3}
    
    Electric Field of a Uniform Ring
    \[
        \vec{E} = E_x\hat{i} = \frac{1}{4\pi\epsilon_0}\frac{Qx}{(x^2+r^2)^{\frac{3}{2}}}\hat{i}
    \]
    
    Electric Field of a Line Segment
    \[
        \vec{E} = \frac{1}{4\pi\epsilon_0}\frac{Q}{x\sqrt{x^2+a^2}}\hat{i}
    \]
    
    Electric Field of a Charged Disk
    \[
        E_x = \frac{\sigma x}{2\epsilon_0}\left[  \frac{-1}{\sqrt{x^2+R^2}}+\frac{1}{x}  \right]
    \]
    
    \end{multicols}
\end{center}
\medskip
\large \textbf{Electric Potential Energy}\normalsize \qquad
Units: J\\
\noindent\rule{\textwidth}{0.5pt}

\begin{center}

    Note: Work of a field, $\Delta U = U_i - U_f$, \underline{not} $U_f - U_i$
    
    \begin{multicols}{3}
        
        EPE Given a Voltage Difference
        \[
            U = q_0V
        \]
        
        EPE in an Electric Field
        \[
            U = q_0Ed
        \]
        
        EPE for Two Point Charges
        \[
            U = \frac{kq_1q_0}{r}
        \]
        
        
    \end{multicols}
    \begin{multicols}{3}
    
    EPE From a Force
        \[
            \int_{r_a}^{r_b}{F_r dr}
        \]
    
    \columnbreak
    
    EPE Produced by a Collection of Point Charges on a Point Charge
    \[
        U = \frac{q_0}{4\pi\epsilon_0} \sum_{i = 1}^{n}{\frac{q_i}{r_i}}
    \]
    
    Collective EPE of a Collection of Point Charges\\
    \footnotesize
    (With Respect to a Specified Origin)\normalsize
    \[
        U = \frac{1}{4\pi\epsilon_0}\sum_{i<j}{\frac{q_iq_j}{r_{ij}}}
    \]
    
    \end{multicols}
\end{center}
\medskip
\large \textbf{Electric Potential/Voltage}\normalsize \qquad
Units: $\frac{\textrm{J}}{\textrm{C}}$ or V\\
\noindent\rule{\textwidth}{0.5pt}
\begin{center}
    Note: Voltage, like electric potential energy is $\Delta V = V_i - V_f$, \underline{not} $V_f - V_i$
\end{center}
\begin{multicols}{5}
\begin{center}
Voltage Produced by a Point Charge
\[
    V = \frac{kq}{r}
\]

\vfill\null\columnbreak

Voltage From EPE
\[
    V = \frac{U}{q_0}
\]

\vfill\null\columnbreak

Voltage of a Collection of Point Charges
\[
    V = k \sum{\frac{q_i}{r_i}}
\]

Voltage of a Charge Distribution
\[
    V = k\int\frac{dQ}{r}
\]

Change in Voltage (Outwards)
\[
    V_{r_2}-V_{r_1} = -\int_{r_1}^{r_2}{\vec{E}\cdot dr}
\]
\end{center}
\end{multicols}
\large \textbf{Torque}\normalsize \qquad
Units: $\textrm{N}\cdot\textrm{m}$\\
\noindent\rule{\textwidth}{0.5pt}
\begin{multicols}{5}

\begin{center}
    
    Electric Dipole Moment
    \[
        p = qd
    \]
    \footnotesize (In the direction of d)\normalsize
    
    \vfill\null\columnbreak
    
    Torque (Scalar)
    \[
        \tau = pE\sin\phi
    \]
    
    \vfill\null\columnbreak
    
    Torque (Vector)
    \[
        \vec{\tau} = \vec{p}\times\vec{E}
    \]
    
    \vfill\null\columnbreak
    
    Potential Energy (Scalar)
    \[
        U = -pE\cos\phi
    \]
    
    \vfill\null\columnbreak
    
    Potential Energy (Vector)
    \[
        U = -\vec{p}\cdot\vec{E}
    \]
    
    \vfill\null
    
\end{center}

\end{multicols}

\vfill\null\pagebreak
\begin{center}

Random
\[
    1\textrm{ eV} = 1.602\cdot10^{-19}\textrm{ J}
\]

Voltage of a Cylinder/Line
\[
    \Delta V_{ab} = \frac{\lambda}{2\pi\epsilon_0}\ln\frac{r_b}{r_a}
\]

Voltage Above a Ring of Charge
\[
    V = \frac{KQ}{\sqrt{x^2+a^2}}
\]

\footnotesize $x$: distance above ring\\
$a$: radius of ring\normalsize\\

Potential of a Line of Charge
\[
    V = \frac{KQ}{2a}\ln\left(\frac{\sqrt{a^2+x^2}+a}{\sqrt{a^2+x^2}-a}\right)
\]

Field of an Infinite Plane 
\[
    E = \frac{\sigma}{2\epsilon_0}
\]

Field of an Infinite Wire
\[
    E = \frac{\lambda}{2\pi r\epsilon_0}
\]

Field of an Infinite Cylinder (Interior)
\[
    E = \frac{\rho r}{2\epsilon_0}
\]

Field of an Infinite Slab (Interior)
\[
    E = \frac{\rho s}{2\epsilon_0}
\]

Field Inside a Sphere
\[
    E = \frac{\rho r}{3\epsilon_0}
\]


\end{center}

\end{document}
