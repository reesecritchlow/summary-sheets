\documentclass{article}
\usepackage[utf8]{inputenc}

\usepackage{summarysheets}

\begin{document}

\begin{center}
    \Large MECH 360 Notes\\
    \normalsize Reese Critchlow
\end{center}

\header{Overview}

There exist four main ways to generate stresses in solid mechanics:
\gap
\sheader{Normal Stress} Normal stress occurs when a force is applied to the normal
face of an object.
\[
    \sigma = \frac{F}{A_\perp}    
\]
\sheader{Shear Stress} Shear stress occurs when a force is applied perpendicular
to the normal face of an object.
\[
    \tau = \frac{V}{A_\parallel}
\]
\sheader{Torsional Stress} Torsional stress occurs when an equitorial moment is applied
about an object. It produces a shear stress.
\[
    \tau = \frac{Tr}{J} = \frac{G\phi}{l}
\]
where $J$ is the polar moment of area and $r$ is the distance from the neutral axis.
\gap
\sheader{Bending Stress} Bending stress occurs when an axial moment is applied about
an object. It produces a normal stress.
\[
    \sigma = \frac{My}{I} = \frac{E}{\rho} 
\]
It is also important to note that $J = I_x + I_y$, so $I = \frac{1}{2} J$.
\gap
\header{Bending Moment Diagrams}
Bending moment diagrams appear to be the heart and soul of solid mechanics. There exists
a standard convention drawing bending moment diagrams:
\begin{enumerate}
    \item Define a coordinate system starting from one of the ends of the beam. Set counterclockwise
    rotation as positive, and downwards shear as positive. 
    \item Make cuts at different points of the beam where different forces behave differently,
    where the ``segment'' that you extract has one end at the coordinate system origin,
    and the other end of the cut.
    \item At the non-origin side of the cut, $p$, draw a counterclockwise moment and a downwards shear
    let the moment be $M$ and the shear be $V$. 
    \item Using the force equilibirum that $\sum M_p = 0$ and $\sum F_y= 0$ for the 
    segment, derive formulae for $M$ and $V$.
    \item Use the formulae for $M$ and $V$ to plot as functions on a graph, and determine
    the max/min values.
    \item Use the fact that $\frac{\partial M}{\partial x} = V$ to check your work.
\end{enumerate}

\header{Beam Deflection}
The deflection of a beam can be calculated using the following formula:
\[
    \frac{d^2y}{dx^2} = \frac{M(x)}{EI}    
\]
However, it is to be noted that this contains a second derivative, so two boundary
conditions are required to solve the problem. Generally, these boundary conditions take
the following forms:
\begin{itemize}
    \item The deflection at a point $a$ is known to be zero $y(a) = 0$.
    \item The deflection at a point $a$ is known to be maximal $y'(a) = 0$.
\end{itemize}

Using the boundary conditions, we can solve the differential equation and solve for
the overall deflection of the beam.
\gap
We can also use superposition to of various deflections to calculate the total deflection.
\gap
\header{Buckling}

For a column/object that is being loaded in compression, it can experience buckling
if the load is large enough. To calculate this, we can follow a similar process as 
beam deflections, with one key difference: we assume there is a $y$ deflection 
such that a moment is induced from the load and the $y$ deflection.

\begin{enumerate}
    \item Define a coordinate system starting from one of the ends of the beam. Set counterclockwise
    rotation to be positive. 
    \item Make a cut at some arbitrary point in the beam, where the ``segment'' starts
    at the origin, and ends at slicing point.
    and the other end of the cut.
    \item At the slicing point of the cut, $p$, draw a counterclockwise moment and a downwards shear
    let the moment be $M$ and the shear be $V$. Assume that the slicing point has 
    a deflection, $y$, which produces a moment.
    \item Using the force equilibirum that $\sum M_p = 0$ for the 
    segment, derive formulae for $M(y)$.
    \item Rearranging the formula from beam deflection: $\frac{d^2y}{dx^2} = \frac{M(x)}{EI}$
    we can obtain:
    \[
        M(y) = EIy''.
    \]
    Thus, we are generally left with an equation of the form:
    \[
        EIy'' + Py = 0
    \]
    \item Given a second order homogenous ODE, one can guess a solution of the form 
    \[
        y(x) = A\cos(\lambda x) + B\sin(\lambda x)
    \]
    \item Solving this differential equation, we can employ the boundary conditions that:
    \begin{enumerate}
        \item $x = 0 \to y = 0$
        \item $x = L \to y = L$
    \end{enumerate}
    \item The solution of this ODE will bring the $P_{\textrm{cr}}$, or the critical
    load that results in bending.
\end{enumerate}

\sheader{Off-Center Loads} In the event of an off-center load, where the load $P$
is off center by a distance $d$, there is a moment $M_0 = Pd$ induced at the top.
This moment causes bending stress in the beam as well. This bending stress is important
for calculating the normal stress for yield strength in the beam, However
\underline{$P_\textrm{cr}$ stays the same}.

\pagebreak
\header{Shearing Stress and Shear Flow}

When a transverse load is applied to a a beam, there exists a shear stress on each
layer \underline{along} of the beam. This is called the ``shear flow'' and is 
denoted by $q$, whose units are force/area. Shear flow is useful, because it can 
allow one to find two important quantities:
\begin{itemize}
    \item Total shearing force on the layer of length $l$: $F = q \cdot l$
    \item Shearing stress at any point on the layer of \textbf{width} $t$: $\tau = \frac{q}{t}$
\end{itemize}  
To calculate $q$, we can use the formula for shear flow.
\begin{multicols}{2}
    \[
        q = \frac{VQ}{I}
    \]
    \columnbreak
    \begin{itemize}
        \item $I$: [Second] Moment of Area with respect to neutral axis
        \item $V$: Shearing force at the section of the beam desired
        \item $Q$: First moment of area, where $Q = (A)(L)$, where $A$ is the 
        area of a section, and $L$ is its distance from the neutral axis.
    \end{itemize}
\end{multicols}

From this concept of shear flow, we can also derive more precise values for the 
maximal shearing stress in various cross sections:
\begin{itemize}
    \item \sheader{Rectangular Cross Sections} $\tau_\textrm{max} = \frac{3}{2} \frac{V}{A}$
    \item \sheader{Cylindrical Cross Sections} $\tau_\textrm{max} = \frac{4}{3} \frac{V}{A}$
    \item \sheader{Thin-Walled Cylindrical Cross Sections} $\tau_\textrm{max} = 2 \frac{V}{A}$
\end{itemize}

\header{Combined Loading}

In the case that an object has both normal and shear stresses in multiple dimensions,
we can define the \underline{stress tensor}, which defines all of the stress at a point
on an object.

\begin{multicols}{2}
    \begin{center}\underline{2D Stress Tensor}\end{center}
    \[
        \begin{bmatrix}
            \sigma_x & \tau_{xy}\\
            \tau_{yx} & \sigma_y
        \end{bmatrix}    
    \]
    \vfill\null\columnbreak
    \begin{center}\underline{3D Stress Tensor}\end{center}
    \[
        \begin{bmatrix}
            \sigma_x & \tau_{xy} & \tau_{xz}\\
            \tau_{yx} & \sigma_y & \tau_{yz}\\
            \tau_{zx} & \tau_{zy} & \sigma_z
        \end{bmatrix}    
    \]
\end{multicols}
It is important to note that the eigenvalues of the stress tensors correspond to 
\underline{principal stresses} of the point of interest. The principal stresses
are the maximum normal stresses of the point of interest. 
\gap
If one orders the stresses from smallest to largest, we can define the maximum shearing
stress as:
\[
    \tau_\textrm{max} = \frac{\sigma_1 - \sigma_3}{2}
\]

\header{Pressure Vessels}

\begin{align*}
    \sigma_\textrm{hoop} = \frac{Pr}{t} && \sigma_\textrm{axial} = \frac{Pr}{2t}
\end{align*}

\pagebreak

\header{Failiure Criteria}
We can also define two types of failiure criteria for a body:
\gap
\sheader{Von Mises} Failiure happens if $\sigma_v$ reaches $S_y$:
\begin{align*}
    f_s = \frac{S_y}{\sigma_V} && \sigma_V = \sqrt{\frac{(\sigma_1 - \sigma_2)^2 + (\sigma_1 - \sigma_3)^2 + (\sigma_2 - \sigma_3)^2}{2}} 
\end{align*}

\sheader{Tresca} Failiure happens if $\tau_\textrm{max}$ reaches $0.5S_y$:
\[
    f_s = \frac{0.5S_y}{\tau_\textrm{max}}    
\]

\section*{Final Exam}

\header{Castigliano's Theorem}

Castigliano's theoerm originates from the existence of strain energy within a system.
Since a form of Hooke's law exists for every single type of strain, whether that be 
axial, shear, bending, or torsional loading. Hence, we can write the energy in a 
system as $U = \frac{1}{2}kx^2$ where $k$ is generally a modulus of some sort, and $x$
is the strain. Hence, we can accumulate all of this strain energy over a volume usin 
integration. Hence, we define four different formulae for such energy:\\
{\renewcommand{\arraystretch}{2}%
\begin{tabular}{ | c | c | }
    \hline
    Axial Loading & $\displaystyle U = \int_0^l \frac{P^2}{2AE}dx$\\
    \hline 
    Bending & $\displaystyle U = \int_0^l \frac{M^2}{2EI}dx$\\
    \hline 
    Torsion & $\displaystyle U = \int_0^l \frac{T^2}{2GJ}dx$\\
    \hline
    Shearing & $\displaystyle U \approx 0$\\
    \hline
\end{tabular}}\gap
With these definitions, then we can introduce the most important part of Castigliano's theorem:
the notion that the displacement of any point $j$ on a structure, subjected to a load $P_j$,
\underline{measured along the line of action of $P_j$} can be expressed as the partial 
derivative of the strain energy of the strucutre with respect to the load $P_j$:
\begin{align*}
    \delta_J = \frac{\partial U}{\partial P_J}
\end{align*}
Hence, it often occurs that a load is not actually applied at a point, thus, we often apply
an ``imaginary'' load at a point that agrees with a system, such that we can take the 
derivative at the end, which will send a lot of things to zero.
\gap
It is also important to take the infinitesimal quantities of the strain at each point.
\gap
\header{Impact Loading}
Given some loading with an initial height $h$ and weight $W$, we can define the effective load $P$
in terms of an impact factor $f_i$:
\begin{align*}
    P &= W \left(1 + \sqrt{1 + \frac{2h}{\delta_{\textrm{st}}}}\right) & P &= Wf_i\\
    f_i &= 1 + \sqrt{1 + \frac{2h}{\delta_\textrm{st}}}
\end{align*}

\pagebreak

\header{Shear Center}

Finding shear center is strange. Here's my general outline thusfar:
\begin{enumerate}
    \item Identify the relevant shear forces in the object under the applied load.
    \item Find the expression for the shear flow in the member.
    \item Divide the shear flow by the thickness to obtain the shear stress.
    \item Multiply the shear stress by the area, $dA$ to obtain shear force.
    \item Integrate the shear force times the distance to a chosen point (colinear with the point force application) 
    through the member to obtain the entire shear force. 
    \item Set $Ve = \int Fr\cdot da$ to solve for $e$.
\end{enumerate}

\end{document}