\documentclass{article}
\usepackage[utf8]{inputenc}
\usepackage[letterpaper, portrait, margin=1in]{geometry}
\usepackage{multicol}
\usepackage{amsmath}
\usepackage{amssymb}
\usepackage{amsthm}
\usepackage{enumerate}
\setlength\parindent{0pt}
\usepackage{enumerate}
\usepackage{graphicx}
\graphicspath{ {./images/} }
\usepackage{fancyhdr}
\usepackage{tcolorbox}
\hyphenchar\font=-1
\usepackage{tabularx}

\newcommand{\header}[1]{\begin{large}\noindent #1\end{large}\\\rule{\textwidth}{0.5pt}}
\newcommand{\sheader}[1]{\underline{#1:}}
\newcommand{\mheader}[1]{\textbf{{\sheader{#1}}}}
\newcommand{\gap}{\medskip\\}
\newcommand{\centertext}[1]{\begin{center}#1\end{center}}
\newcommand{\bfrac}[2]{\left(\frac{#1}{#2}\right)}
\newcommand{\formula}[3]{\begin{center} \begin{tcolorbox}[title = #2] $$#3$$\end{tcolorbox}\end{center}}
\newcommand{\where}{\hspace{0.5cm} \textrm{where} \hspace{0.5cm}}
\newcommand{\hgap}{\hspace{0.5cm}}
\newcommand{\pfrac}[2]{\frac{\partial #1}{\partial #2}}
\newcommand{\doubleformula}[4]{\begin{center} \begin{tcolorbox}[title = #2] $$#3$$\\$$#4$$\end{tcolorbox}\end{center}}
\newcommand{\curly}[1]{\left\{#1\right\}}
\newcommand{\proj}[2]{{}\textrm{proj}_{#1}\left(#2\right)}
\newcommand{\sgap}{\smallskip\\}

\newcommand{\ds}{\displaystyle}
\newcommand{\Arg}{\textrm{Arg}}
\newcommand{\tabitem}{~~\llap{\textbullet}~~}


\newcommand{\tripleformula}[5]{\begin{center} \begin{tcolorbox}[title = #2] $$#3$$\\$$#4$$\\$$#5$$\end{tcolorbox}\end{center}}

\usepackage{physics}
% \usepackage{braket}

\begin{document}
\begin{center}
        \Large MATH 400 Cookbook\\
        \normalsize Reese Critchlow\\
        \textit{Any Distribution for Commercial Use Without the Expressed Consent of the Creator is Strictly Forbidden.}
\end{center}

\header{First Order PDEs}

Given some PDE of the form:

\begin{align*}
    \begin{cases}
        u_t + c(x, t, u)u_x = s(x, t, u)\\
        u(x, 0) = f(x)
    \end{cases}
\end{align*}
We solve by the following:
\begin{enumerate}
    \item Obtain the system of ODEs corresponding to the original PDE:
    \begin{align*}
        \frac{dx}{dt} &= c(x, t, u) && \frac{du}{dt} = s(x, t, u).
    \end{align*}
    \item Solve whichever equation has no variables that do not appear in derivative term.
    \item Depending on which equation was solved in step 2, impose the corresponding initial conditions:
    \begin{align*}
        x(0) &= \xi && u(x, 0) = f(\xi).
    \end{align*}
    \item Use the result of step 3 to substitute back into the incomplete equation.
    \item Solve for $\xi$, and substitute it back into the result for $u$.
\end{enumerate}

\sheader{Alternate Strategy}
\gap
Given some PDE of the form:
\begin{align*}
    \begin{cases}
        a(x, t, u)u_t + b(x, t, u)u_x = c(x, t, u)\\
        u(x, 0) = f(x) \textrm{ on } t = g(x).
    \end{cases}
\end{align*}

We solve the following system of ODEs:
\begin{align*}
    \frac{dt}{ds} &= a(x, t, u) && t(0) = g(\xi)\\
    \frac{dx}{ds} &= b(x, t, u) && x(0) = \xi\\
    \frac{du}{ds} &= c(x, t, u) && u(x=0) = f(\xi)
\end{align*}

\sheader{Data Curves and Characteristic Curves}
Through solving these PDEs, we obtain the following:
\begin{itemize}
    \item Data Curve: the intitial condition $t = g(x)$ is called the \textit{data curve},
    as it contains the original data from the IC.
    \item Characteristic Curves: the characteristic curves are given by $\xi = h(x, y)$.
    They describe the curves on the space where the PDE reduces to the ODE $\frac{du}{dt} = 0$.
\end{itemize}

There exist some complications when special cases of the data curves and the 
characteristic curves:
\begin{itemize}
    \item Two Characteristics Intersect
    \begin{itemize}
        \item Cannot solve $s(x, y), \xi(x, y)$.
        \item $u$ may be multi-valued.
    \end{itemize}
    \item Characteristic Intersects the Data Curve Twice
    \begin{itemize}
        \item No solution in general.
    \end{itemize}
    \item Data Curve is a Characteristic Curve
    \begin{itemize}
        \item Need to choose some special $u = f(\xi)$ to avoid contradiction,
        and cannot determine $u$ outside the data curve.
        \item $u$ cannot be determined outside the data curve.
    \end{itemize}
\end{itemize}

\header{Breaking}

For some problems, we get that multiple characteristics intersecting at some 
certain point. The first point at which characteristics intersect is called 
the ``breaking point''. This generally is most seen in PDEs of the form:
\begin{align*}
    \begin{cases}
        u_t + uu_x = 0\\
        u(x, 0) = \phi(x)
    \end{cases}
\end{align*}
Breaking [allegedly] occurs when $u_x \to \infty$, so to find this, we take
\begin{align*}
    \frac{d}{dx} x(\xi, t),
\end{align*}
and solve for the condition which makes $\xi_x \to \infty$. 
\gap
\header{Traffic Flow Problems}
Traffic flow problems arise from integral conservation laws. They assume the 
general form:
\begin{align*}
    \begin{cases}
        \rho_t + c(\rho) \rho_x = 0\\
        \rho(x, 0) = f,
    \end{cases}
\end{align*}
where $f$ is generally a piecewise constant function. For demonstration, let 
\begin{align*}
    f(x) = \begin{cases}
        A, & x < a\\
        B, & a < x < b\\
        C, & x > b
    \end{cases}
\end{align*}

\sheader{General Solution Strategy}

\begin{enumerate}
    \item Convert the PDE into a system of ODEs:
    \begin{align*}
        \frac{dx}{dt} = c(\rho) && \frac{d\rho}{dt} = 0.
    \end{align*}
    The latter of the two equations implies that $\rho = \textrm{constant}$,
    and with the condition that $u(x, 0) = f$, then $u(x, t) = f$. However, 
    we still need to re-translate the bounds of the piecewise function.
    \item Hence, we then solve the first equation in the ODE with initial condition
    $x(0) = x_0 = \xi$ to obtain the form:
    \begin{align*}
        x(t, \xi) = c(\rho)t + \xi.
    \end{align*}
    Actually, it is more elegant to view this in terms of $x_0$.
    \begin{align*}
        x(t, x_0) = c(\rho(x_0))t + x_0
    \end{align*}
    We can then use this to readjust all of the terms in the piecewise function. 
    Looking back at $f(x)$, since these points are all at $t = 0 \implies x = x_0$,
    then, we can rewrite the bounds of $f$:
    \begin{align*}
        x<a &\implies x_0 < a\\
        &\implies x < c(\rho(a))t + a
    \end{align*}
    Hence, we get a full solution with these results, and so on.
    \item Lastly, it is often the case that there are discontinuities in the result 
    $u$. There are two cases in which this happens:
    \begin{enumerate}
        \item \sheader{Fanlike Characteristics} In this case,
        one will end up with a solution where there are regions with undefined 
        values in the solution. To find the full solution, we do the following:
        \begin{enumerate}
            \item Identify the points of $(x_0)_i$ from where the discontinuity
            arises from. (i.e. $a$, $b$)
            \item Sub in $(x_0)_i$ into the $x$ equation and solve for $f(x_0) = \rho((x_0)_i)$:
            \begin{align*}
                x = c(\rho((x_0)_i))t + x_0.
            \end{align*}
        \end{enumerate}
        \item \sheader{Shock Characteristics} In this case, one will end up with a solution
        where characteristics cross. This also corresponds to solutions which are
        piecewise and boundaries overlap. In these cases one must do the following:
        \begin{enumerate}
            \item Determine the shock point, which is the point where the shock 
            first occurs $(s_0)$.
            \item Determine the shock function, which can be determined by the 
            following equation:
            \begin{align*}
                \frac{ds}{dt} = \frac{Q(\rho_+)  - Q(\rho_-)}{\rho_+ - \rho_-},
            \end{align*}
            where $q_x(\rho) = \rho_x c(\rho)$ (generally have to solve that before).
        \end{enumerate}
    \end{enumerate}
\end{enumerate}

\header{Second Order Linear PDEs}
\sheader{Well-posed-ness} There are three conditions that need to hold for an Initial Boundary Value Problem 
(IBVP) for a PDE that need to hold in order for it to be well-posed. They 
are as follows:
\begin{enumerate}
    \item \sheader{Existence} There exists at least one solution $u(x, t)$,
    satisfying all conditions.
    \item \sheader{Uniqueness} There exists at most one solution $u(x, t)$,
    satisfying all conditions.
    \item \sheader{Stability} The solution depends in a stable manner on the 
    initial data, that is if the data changes by a small amount, then the solution 
    changes by only a small amount as well.
\end{enumerate}
\sheader{Methods for Proving Stability}
\begin{enumerate}
    \item Let $u(x, t)$ be a solution for the IBVP with data $f(x)$. Let 
    $\Tilde{u}(x, t)$ be a solution for the IBVP with data $\Tilde{f}(x)$.
    Show that if $\max_x |f(x) - \Tilde{f}(x)|$ is bounded by a small amount, 
    then $\max_x |u(x, t) - \Tilde{u}(x, t)|$ is also bounded by a small amount,
    for any $t > 0$.
    \item If a sequence of inital data of the form $f_n(x)$ has that 
    $\max_x |f_n(x)| \to 0$ as $n\to \infty$, then prove that $\max_x|u_n(x, t)| \to 0$,
    as $n\to \infty, \,\, \forall t > 0$.
    
\end{enumerate}

\header{Classifications and Transformations of PDEs} Second Order Linear PDEs 
of the following form:
\begin{align*}
    au_{xx} + 2bu_{xy} + cu_{yy} + du_x + eu_y + fu = g(x, y)
\end{align*}
with \underline{determinant} $D = b^2 - ac$ can be classified into the following forms:
\begin{enumerate}
    \item Elliptic $(D < 0)$:
    $
        \ds u_{xx} + u_{yy} = \textrm{Lower Order Terms}
    $
    \item Hyperbolic $(D > 0)$:
    $\ds
        u_{xx} - u_{yy} = \textrm{Lower Order Terms}
    $
    \item Parabolic $(D = 0)$:
    $\ds
        u_{xx} = \textrm{Lower Order Terms}
    $
\end{enumerate}
\pagebreak
\sheader{General Form for Classifying and Transforming PDEs}
\begin{enumerate}
    \item Classify the PDE using the determinant.
    \item Rewrite the PDE into an operator form, i.e.:
    \begin{align*}
        u_{xx} + 8u_{xy} = (\partial_x^2 + 8 \partial_x \partial_y) u.
    \end{align*}
    \item Complete the square with the partials.
    \item Substitute operators such that the PDE takes the classified form, with 
    new variables.
    \item With the coordinate transform from $x, y \to \xi, \eta$, 
    write the matrix $B^T$, which fits the following form:
    \begin{align*}
        \begin{pmatrix}
            \partial_x\\
            \partial_y
        \end{pmatrix}
        = B^T
        \begin{pmatrix}
            \partial_\xi\\
            \partial_\eta
        \end{pmatrix}.
    \end{align*}
    \item Use the matrix $B$ to obtain $\xi$ and $\eta$ in terms of $x, y$:
    \begin{align*}
        \begin{pmatrix}
            \xi\\\eta
        \end{pmatrix} = B
        \begin{pmatrix}
            x\\y
        \end{pmatrix}
    \end{align*}
\end{enumerate}

\header{Wave Equation}

\sheader{General Form Solution}
\begin{align*}
    \begin{cases}
        u_{tt} - c^2 u_{xx} = f(x, t)\\
        u(x, 0) = u_0(x)\\
        u_t(x, 0) = u_1(x).
    \end{cases}
\end{align*}
For a problem of this type, the solution is of the form:
\begin{align*}
    u(x, t) = \frac{1}{2} \left[u_0(x+ct) + u_0(x - ct)\right] + \frac{1}{2c} \int_{x-ct}^{x + ct} u_1(y)dy
    + \frac{1}{2c} \int_{0}^{t}\int_{x-c(t-s)}^{x+c(t-s)}f(y, s)dyds.
\end{align*}
\raggedleft\qedsymbol\\
\raggedright
There are other forms of the wave equation for which the solution is 
not as trivial.\gap
\sheader{Constant BC}
\begin{align*}
    \begin{cases}
        u_{tt} - c^2 u_{xx} = 0, &  x > 0, t > 0\\
        u(x, 0) = u_0(x), \,\, u_t(x, 0) = u_1(x), &  x > 0\\
        u(0, t) = c & t > 0.
    \end{cases}
\end{align*}
To solve this form, we know that the general solution to the wave equation is 
always:
\begin{align*}
    u(x, t) = f(x + ct) + g(x - ct)
\end{align*}
Hence, we can set up two systems of equations to solve:
\begin{align*}
    \begin{cases}
        u(x, 0) = u_0(x) = f(x) + g(x)\\
        u_t(x, 0) = u_1(x) = cf'(x) + c'g(x)
    \end{cases}
    &&
    \textrm{for}\,\, x \geq ct\\
    \begin{cases}
        u(x, 0) = u_0 = f(x) + g(x)\\
        u(0, t) = c = f(ct) + g(-ct) = f(ct=x) + g(-ct = -x) 
    \end{cases} && 
    \textrm{for} \,\, 0 < x < ct.
\end{align*}

\pagebreak

\sheader{Factored Wave Equation}

Given some wave equation of the form:
\begin{align*}
    \begin{cases}
        (a\partial_x + b\partial_y)(c\partial_x - d\partial_y) u = 0,\\
        u(x, 0) = u_0(x),\\
        u_t(x, 0) = e^x
    \end{cases}
\end{align*}
The general solution reduces to:
\begin{align*}
    u(x, t) = f(ax + bt) + g(cx - dt).
\end{align*}
Which can be solved using a system of equations similar to the prior case.
\gap
\sheader{Energy and Waves}
Since energy in a system should be conserved, we can calculate the total 
energy of a wave PDE with the general form:
\begin{align*}
    \rho u_{tt} = Tu_{xx},
\end{align*}
the energy is given by:
\begin{align*}
    E = \frac{1}{2} \int_{-\infty}^{\infty} \rho u_t^2(x, 0) + Tu_x^2(x, 0) dx
\end{align*}

\sheader{Method of Reflections}

For equations of the form:
\begin{align*}
    \begin{cases}
        u_{tt} - c^2 u_{xx} = f(x, t), & x > 0\\
        u(x, 0) = u_0(x), & x > 0\\
        u_t(x, 0) = u_1(x), & x > 0
    \end{cases}
\end{align*}

The solution is of the form:
\begin{align*} u(x, t) = 
    \begin{cases}
        \frac{1}{2}\left[u_0(x +ct) + u_0(x-ct)\right] + \frac{1}{2c}\int_{x-ct}^{x+ct}u_1(y)dy
        + \frac{1}{2c}\int_{0}^{t}\int_{x-c(t-s)}^{x+c(t-s)} f(y, s)dyds, & x > ct\\
        \frac{1}{2}\left[u_0(x +ct) + u_0(x-ct)\right] \\ + \frac{1}{2c}\left(\int_{x-ct}^{0}u_1^{\textrm{ext}}(y)dy + \int_{0}^{x+ct}u_1^{\textrm{ext}}(y)dy\right)
        \\+ \frac{1}{2c}\left(\int_{0}^{t - \frac{x}{c}}\int_{-x+c(t-s)}^{x+c(t-s)} f^{\textrm{ext}}(y, s)dyds + \int_{t-\frac{x}{c}}^{t}\int_{x-c(t-s)}^{x+c(t-s)} f^{\textrm{ext}}(y, s)dyds \right), & 0 < x < ct
    \end{cases}
\end{align*}

In the case that the equation has the form: 
\begin{align*}
    \begin{cases}
        u_{tt} - c^2 u_{xx} = f(x, t), & x > 0 \\
        u(x, 0) = u_0(x), & x > 0\\
        u_t(x, 0) = u_1(x), & x > 0\\
        u(0, t) = h(t)
    \end{cases}
\end{align*}

Then it is the same as the standard method of reflections, just with the addition 
of an extra term:
\begin{align*}
    u(x, t) = u_0 \textrm{ term } + u_1 \textrm{ term } + h\left(t - \frac{x}{c}\right) + \frac{1}{2c} \iint_\Delta f.
\end{align*}

\sheader{Reflections on the Finite Interval} For reflections on the finite interval,
we take the odd periodic extension of all functions and proceed the same as prior.
Generally, questions of this nature will ask for a value at a point instead of 
a general solution.

\pagebreak

\header{Heat Equation}
\sheader{General Form} Given a heat equation of the form:
\begin{align*}
    \begin{cases}
        u_t = k u_{xx},\\
        u(0, t) = u_0(x),
    \end{cases}
\end{align*}
a solution to the heat equation is:
\begin{align*}
    u(x, t) = \int_{-\infty}^{\infty}S(x - y, t)u_0(y)dy.
\end{align*}

\sheader{Heat Equation with a Source} Given a heat equation of the following form:
\begin{align*}
    \begin{cases}
        u_t = ku_{xx},\\
        u(0, t) = u_0(x),
    \end{cases}
\end{align*}
a solution to the heat equation is:
\begin{align*}
    u(x, t) = \int_{-\infty}^{\infty} S(x - y, t)u_0(y)dy + \int_{0}^{t}\int_{-\infty}^{\infty} S(x -y, t-s)f(y, s)dyds.
\end{align*}

\sheader{Well-posedness} Given a heat equation on the whole line:
\begin{align*}
    \begin{cases}
        u_t = ku_{xx},\\
        u(0, t) = u_0(x),
    \end{cases}
\end{align*}
it follows that the following criteria must hold for a IBVP to be \textit{well-posed}:
\begin{enumerate}
    \item Existence: There is at least one solution.
    \item Uniqueness: The solution is unique and bounded.
    \item Stability: We prove stability by the following method: let
    \begin{align*}
        M = \max_{y \in \mathbb{R}} |u_0(y)|, && F = \max_{y \in \mathbb{R}, 0 \leq s \leq T} |f(y, s)|,
    \end{align*}
    then, it follows that:
    \begin{align*}
        \max_{0 \leq t \leq T, x \in \mathbb{R}}|u(x, t)| \leq M + FT.
    \end{align*}
    This implies stability, such that if $u_0 - \Tilde{u_0}$ and $f - \Tilde{f}$ are small, then $u-\Tilde{u}$ is also small, as:
    \begin{align*}
        \max_{x \in \mathbb{R}, \,\, 0 \leq t \leq T} |u(x, t) - \Tilde u(x, t) | \leq \max_{x \in \mathbb{R}} |u_0(x) - \Tilde{u_0}(x)| 
        + T \max_{x\in \mathbb{R}, \, 0 \leq t \leq T}|f(x, t) - \Tilde{f}(x, t) |.
    \end{align*}
\end{enumerate}

\sheader{Maximal Principle for Heat Equation} If some $u(x,t)$ satisfies the heat 
equation in a rectangle, then the maximum value of $u(x, t)$ is assumed either 
initially or on the lateral sides ($x=0$ or $x=l$).
\gap
\sheader{Comparison Principle} If, for the domain $U = (a, b) \times (0, T)$,
then if 
\begin{align*}
   u_t \leq ku_{xx} \textrm{ (subsolution)}, && v_t \geq ku_{xx} \textrm{ (supersolution)}
\end{align*}
and $u \leq v$ on $\partial U$, then
\begin{align*}
    u(x,t) \leq v(x, t)
\end{align*}
for all of $U$.


\end{document}