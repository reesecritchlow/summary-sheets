\documentclass{article}
\usepackage[utf8]{inputenc}
\usepackage[letterpaper, portrait, margin=1in]{geometry}
\usepackage{multicol}
\usepackage{amsmath}
\usepackage{amssymb}
\usepackage{enumerate}
\setlength\parindent{0pt}
\usepackage{enumerate}
\usepackage{graphicx}
\graphicspath{ {./images/} }
\usepackage{fancyhdr}
\usepackage{tcolorbox}
\hyphenchar\font=-1
\usepackage{tabularx}

\newcommand{\header}[1]{\begin{large}\noindent #1\end{large}\\\rule{\textwidth}{0.5pt}}
\newcommand{\sheader}[1]{\underline{#1:}}
\newcommand{\mheader}[1]{\textbf{{\sheader{#1}}}}
\newcommand{\gap}{\medskip\\}
\newcommand{\centertext}[1]{\begin{center}#1\end{center}}
\newcommand{\bfrac}[2]{\left(\frac{#1}{#2}\right)}
\newcommand{\formula}[3]{\begin{center} \begin{tcolorbox}[title = #2] $$#3$$\end{tcolorbox}\end{center}}
\newcommand{\where}{\hspace{0.5cm} \textrm{where} \hspace{0.5cm}}
\newcommand{\hgap}{\hspace{0.5cm}}
\newcommand{\pfrac}[2]{\frac{\partial #1}{\partial #2}}
\newcommand{\doubleformula}[4]{\begin{center} \begin{tcolorbox}[title = #2] $$#3$$\\$$#4$$\end{tcolorbox}\end{center}}
\newcommand{\curly}[1]{\left\{#1\right\}}
\newcommand{\proj}[2]{{}\textrm{proj}_{#1}\left(#2\right)}
\newcommand{\sgap}{\smallskip\\}

\newcommand{\ds}{\displaystyle}
\newcommand{\Arg}{\textrm{Arg}}
\newcommand{\tabitem}{~~\llap{\textbullet}~~}


\newcommand{\tripleformula}[5]{\begin{center} \begin{tcolorbox}[title = #2] $$#3$$\\$$#4$$\\$$#5$$\end{tcolorbox}\end{center}}

\usepackage{physics}
% \usepackage{braket}

\begin{document}
\begin{center}
        \Large MATH 318 Review Notes\\
        \normalsize Reese Critchlow
\end{center}


\header{Basic Concepts in Probability}

\sheader{Permutations} For some set $S = \curly{a, b, c}$, the number of orderings of the set, or 
the number of \textit{permutations} of the set is given by the \underline{factorial}, that is, that
the number of permutations for a set of distinct elements of size $n$ is $n!$.

\sheader{Redundancies in Permutations} It is important to note, that in a set of size $n$ where there are $m$ identical/redundant
elements with count $m_i$, the number of permutations is given by:

\begin{align*}
    \frac{n!}{\prod_{i=1}^{m} m_i!}
\end{align*}

\sheader{Choosing and the Binomial Coefficient} The number of ways to choose $k$ unique objects from a set of size
$n$ is given by the \underline{binomial coefficient}:
\begin{align*}
    \binom{n}{k} = \frac{n(n-1)(n - 1) \cdots (n - k + 1)}{k!} = \frac{n!}{k!(n-k)!}
\end{align*}

\sheader{Multinomial Coefficient} Let $b_1, \ldots, b_k$ be nonnegative integers, and let $n = b_1 + b_2 + \cdots b_k$.
The multinomial coefficient $\binom{n}{b_1, b_2, \ldots, b_k}$ describes:
\begin{itemize}
    \item The number of ways to put $n$ interchangeable objects into $k$ boxes, such that box $i$ has $b_i$ 
    objects in it, for $1 \leq i \leq k$.
    \item The number of ways to choose $b_1$ interchangeable objects from $n$ objects, then $b_2$ from the remaining,
    until you choose $b_{k-1}$ from what remains.
    \item The number of unique permutations of a word with $n$ letters $k$ distinct letters, such that the
    $i$-th letter occurs $b_i$ times.
\end{itemize}
The multinomial coefficient can be given by:
\begin{align*}
    \frac{n!}{b_1!b_2!\cdots b_k!}
\end{align*}

\sheader{Mathematical Definition of Probability}
A \textit{probability} is a function that assigns to each $E$ contained in $S$, a number $P(E)$ such that:
\begin{enumerate}
    \item $0 \leq P(E) \leq 1$, $E \subseteq S$
    \item $P(S) = 1$
    \item $E_i \cap E_j = \emptyset, \forall i \neq j \implies P(E_1 \cup E_2 \cup \cdots E_n) = P(E_1) + P(E_2) + \cdots + P(E_n)$    
\end{enumerate}

\sheader{Basic Terminiology}
\begin{itemize}
    \item \textbf{Sample Space}, $S$, describes the set of all possible outcomes of an experiment.
    It is either continuous or discrete, finite or infinite.
    \item \textbf{Event}, $E$, is a subset of the sample space $E \subseteq S$.
    \item A \textbf{Probability Space} is a triplet containing a sample space $S$, a set of events $E$,
    and a probability function $P$, that is $(S, E, P)$.
    \begin{itemize}
        \item Often is the face that $S$ is finite, and every outcome is equally likely, such that \\
        $P(E) = \frac{\textrm{\# of outcomes in } E}{\textrm{\# of outcomes in } S}$
    \end{itemize}
\end{itemize}

\sheader{Properties of Probability}
\begin{enumerate}
    \item $\forall E, P(E) + P(E^C) = P(S) = 1$
\end{enumerate}

\pagebreak

\sheader{Inclusion-Exclusion Formula} For a union of non-disjoint events, the probability of the union is given:
\begin{align*}
    P(E_1 \cup \cdots \cup E_n) = &\sum_{i = 1}^{n} P(E_i) - \sum_{i<j}P(E_i \cap E_j) + \\ 
    &\sum_{i < j < k}P(E_i \cap E_j \cap E_k) - \cdots + (-1)^{n -1}\sum_{i < \cdots}P(E_1 \cap \cdots \cap E_n)
\end{align*}

\sheader{Conditional Probability}
Two events $E$ and $F$ are said to be independent events if:
\begin{align*}
    P(E \cap F) = P(E)P(F) \iff P(E|F) = P(E)
\end{align*}
Hence, it is important to define the notation for a conditional probability:
\begin{align*}
    P(E|F)
\end{align*}
describes the probability of ``$E$ given $F$''.
\sgap
To generalize, events $E_1, \ldots, E_n$ are said to be independent if:
\begin{align*}
    P(E_i \cap \cdots \cap E_n) = P(E_1) \cdots P(E_n).
\end{align*}
Finally we can give some formulae:
\begin{align*}
    P(E \cap F) = P(E|F)P(F) && P(E|F) = \frac{P(E \cap F)}{P(F)}\\
    P(A \cap B) = P(A|B)P(B) = P(B|A)P(A)
\end{align*}

\sheader{Law of Total Probability} Let $F_1, \cdots, F_n$ be a partition of $S$. Then,
\begin{align*}
    P(E) = \sum_{i = 1}^{n} P(E|F_i)P(F_i)
\end{align*}

\sheader{Bayes Formula} Building off of the law of total probability:
\begin{align*}
    P(F_j|E) = \frac{P(E|F_j)P(F_j)}{\sum_{i=1}^{n}P(E|F_i)P(F_i)}
\end{align*}

\header{Random Variables}

\sheader{Definition} A random variable is a function that takes an event space $S$
and produces an event.
\begin{align*}
    X: S \to E
\end{align*}
The event $E$ is often a real number $\mathbb{R}$.
\gap
\header{Discrete Random Variables} Discrete random variables are random variables that take on values in a countable set.
\gap
\sheader{Probability Mass Function} The probability mass function (p.m.f.) $P: E \to \mathbb{R}$ associated with a discrete 
random variable defines the probability that the random variable takes certain values.

\begin{align*}
    P(E) = P(X=E)
\end{align*}

\sheader{Note} it is required that $\sum_i p(x_i) = 1$.
\gap

\sheader{Notation} We say that a random variable $X$ is distributed in a certain way with the notation $X \sim \textrm{dist}(p)$.
\pagebreak

\renewcommand{\arraystretch}{1.2} % Increase vertical spacing

\begin{tabularx}{\textwidth}{|>{\hsize=0.25\hsize\raggedright\arraybackslash}X|>{\hsize=0.75\hsize\raggedright\arraybackslash}X|}
    \hline
    \multicolumn{2}{|c|}{Geometric Random Variable}\\
    \hline
    Interpretation & The probability distribution of $i$ 
    Bernoulli Trials (binary output) with probability $p$ of being successful before one success is obtained.\\
    \hline
    p.m.f. & $p(i) = (1-p)^{i - 1} p$\\
    \hline
    Symbol & $X \sim \textrm{geom}(p)$\\
    \hline
    Notes & Memoryless: $P(X > m + n | X > m) = P(x > n)$\\
    \hline
\end{tabularx} 
\gap
\begin{tabularx}{\textwidth}{|>{\hsize=0.25\hsize\raggedright\arraybackslash}X|>{\hsize=0.75\hsize\raggedright\arraybackslash}X|}
    \hline
    \multicolumn{2}{|c|}{Binomial Random Variable}\\
    \hline
    Interpretation & The binomial random variable describes the distribution of successful Bernoulli
    trials with probability of success $p$ over $n$ trials.\\
    \hline
    p.m.f. & $p(i) = \binom{n}{i}p^i(1-p)^i$\\
    \hline
    Symbol & $X \sim \textrm{Bin}(n,p)$\\
    \hline
\end{tabularx} 
\gap
\begin{tabularx}{\textwidth}{|>{\hsize=0.25\hsize\raggedright\arraybackslash}X|>{\hsize=0.75\hsize\raggedright\arraybackslash}X|}
    \hline
    \multicolumn{2}{|c|}{Poisson Random Variable (with parameter $\lambda > 0$)}\\
    \hline
    Interpretation & Arises as a simplification to the Binomial Random Variable.
    For some parameter $\lambda$, with the expectation of $\lambda$ events in a given interval, it give the probability of $k$
    events occuring in the same interval.\\
    \hline
    p.m.f. & $p(k) = \frac{\lambda^k e^{-\lambda}}{k!}$\\
    \hline
    Symbol & $X \sim \textrm{Bin}(n,p)$\\
    \hline
    Examples & 
        \sheader{Radioactive Decay} For $n$ atoms, each atom has probability $p$ of 
        decaying in a 1 second interval. Thus $\lambda = np$: the average number of 
        decays in 1 second (determined emperically).\\
    \hline
\end{tabularx}
\gap
\header{Continuous Random Variables}
\sheader{Definition} $X$ is a continuous random variable if 
\begin{enumerate}
    \item There exists a function $f$ with $f(x) \geq 0$.
    \item $P(x\in B) = \int_B f(x)dx, \,\, \forall B\subseteq \mathbb{R}, x \in \mathbb{R}$.
\end{enumerate}
In this case $f$ is called the probability density function for the random variable $X$,
and $f(a)$ indicates how likely it is for $x$ to be near $a$, but importantly,
$f(a) \neq P(a)$.
\gap
\sheader{Cumulative Distribution Function} We can also define the Cumulative Distribution Function,
or the c.d.f. for a random variable $X$:
\begin{align*}
    F(a) = P(x \leq a) = P(x \in (-\infty, a]) = \int_{-\infty}^a f(x)dx.
\end{align*}
Importantly, it follows that the relationship between the c.d.f. and the p.d.f. is that
the p.d.f. is the derivative of the c.d.f., that is:
\begin{align*}
    F'(x) = f(x).
\end{align*}
\sheader{Common Continuous Random Variables}

\begin{tabularx}{\textwidth}{|>{\hsize=0.25\hsize\raggedright\arraybackslash}X|>{\hsize=0.75\hsize\raggedright\arraybackslash}X|}
    \hline
    \multicolumn{2}{|c|}{Uniform Random Variable}\\
    \hline
    Interpretation & All values in a range $[a, b]$ are equally likely.\\
    \hline
    p.m.f. & $p(x) = \frac{1}{b - a}$\\
    \hline
    Symbol & $X \sim \textrm{Unif}(a,b)$\\\hline
\end{tabularx}
\gap
\begin{tabularx}{\textwidth}{|>{\hsize=0.25\hsize\raggedright\arraybackslash}X|>{\hsize=0.75\hsize\raggedright\arraybackslash}X|}
    \hline
    \multicolumn{2}{|c|}{Exponential Random Variable}\\
    \hline
    Interpretation & Time of occurrence for an unpredictable event.\\
    \hline
    p.m.f. & $p(x) = \lambda e^{-\lambda x}$\\
    \hline
    Symbol & $X \sim \textrm{Exp}(a,b)$\\\hline
    Notes & 
        \sheader{Memoryless} $P(x > 2\tau\, |\, x > \tau) = P(x > \tau) = \frac{1}{2}$\linebreak\smallskip

        \sheader{Half Life, etc} Exponential RVs are good for modelling exponential decay.
        We can then find the half life, such that $P(x > \tau) = \frac{1}{2}$, which 
        is the time where it is equally likely for the material to have decayed than for 
        it to have not. 
    \\\hline
\end{tabularx}
\gap
\begin{tabularx}{\textwidth}{|>{\hsize=0.25\hsize\raggedright\arraybackslash}X|>{\hsize=0.75\hsize\raggedright\arraybackslash}X|}
    \hline
    \multicolumn{2}{|c|}{Gaussian Random Variable}\\
    \hline
    Interpretation & Unsure\\
    \hline
    p.m.f. & $\ds f(x) = \frac{1}{\sqrt{2\pi} \sigma} e^{\frac{-(x-\mu)^2}{2\sigma^2}}$\\
    \hline
    Symbol & $X \sim \textrm{N}(\mu, \sigma^2)$\\\hline
    Notes & 
        \sheader{Scaling Property} If $X$ is normal $(X \sim N(\mu, \sigma^2))$,
        then $Y \sim N(0, 1)$, where $Y = \frac{X - \mu}{\sigma}$, and $X$ is the 
        value of the random variable. An example of this would be, if one is trying to 
        find the probability that $X \in [a, b]$, with $X \sim N(\mu, \sigma^2)$, then
        the equivalent form would be to find the probability that $Y \in [\frac{a - \mu}{\sigma}, \frac{b - \mu}{\sigma}]$.
    \\\hline
\end{tabularx}
\gap
\header{Expectation Value}

We can define the \underline{expectation value} $\mathbb{E}$ for a continuous random variable $X$ as:
\begin{align*}
    \mathbb{E}X = \int_{-\infty}^{\infty} xf(x)dx.
\end{align*}
Similarly, for a discrete random variable, we have:
\begin{align*}
    \mathbb{E}X = \sum_{\textrm{all } i} k_ip(k_i)
\end{align*}
\sheader{Lemma} Suppose $X$ is a continuous RV with p.d.f. $f$ and $f(x) = 0$ $\forall x < 0$,
then:
\begin{align*}
    \mathbb{E}X = \int_0^\infty P(X > x)dx.
\end{align*}
Similarly, for a discrete RV:
\begin{align*}
    \mathbb{E}X = \int_{n = 0}^\infty P(X > n).
\end{align*}
\header{Compositions of Random Variables}
\sheader{Theorem (Law of the Unconcious Statistician)} For a continuous RV $X$ with 
p.d.f. $f(x)$ and a function $g: \mathbb{R} \to \mathbb{R}$, then:
\begin{align*}
    \mathbb{E}g(X) = \int g(x)f(x) dx
\end{align*}
This can also be applied to discrete RVs:
\begin{align*}
    \mathbb{E}g(X) = \sum_i g(x_i) p(x_i)
\end{align*}
\sheader{Linearity of Expectation}
Expectation values are linear, that is:
\begin{align*}
    \mathbb{E}(aX + b) = a\mathbb{E}X + b
\end{align*}

\header{Moments of Random Variables and Related Values}
We can define the \underline{$\textrm{n}^\textrm{th}$ moment} of $X$ to be:
\begin{align*}
    \mathbb{E}X^n = \int_{-\infty}^{\infty}x^n f(x) dx
\end{align*}
or, in the discrete case:
\begin{align*}
    \sum_i x_i^n p(x_i).
\end{align*}

\sheader{Variance} We define variance as:
\begin{align*}
    \textrm{Var}(X) = \sigma^2 &=  \mathbb{E} \left[(X - \mathbb{E}X)^2\right]\\
    &= \mathbb{E}(X^2) - \left(\mathbb{E}(X)\right)^2.
\end{align*}
\sheader{Standard Deviation} Is the square root of the variance:
\begin{align*}
    \sigma = \sqrt{\textrm{Var}(x)}
\end{align*}

\header{Joint Distributions}

Discrete random variables $X, Y$ have a joint p.m.f. 
\begin{align*}
    p(x, y) = P\left(\curly{X = x} \cap \curly{Y = y}\right).
\end{align*}
They also have \underline{marginal p.m.f.s}:
\begin{align*}
    P_X(x) = \sum_y p(x, y) = P(X = x) && P_Y(y) = \sum_x p(x, y) = P(Y = y).
\end{align*}

In the continuous case, $X$ and $Y$ are jointly continuous with p.m.f. $f(x, y)$ if $P((X, Y) \in C) = \iint_C f(x, y)dxdy$.
Thus, we can also define the marginal p.m.f. for continuous distributions:
\begin{align*}
    P(x \in A) = P(X \in A, y \in \mathbb{R}) = \int_{D_X} f(x, y)dy
\end{align*}
where $D_X$ is the domain of $X$ for a given $y$. And thus, similarly,
\begin{align*}
    P(y \in B) = \int_{D_Y} f(x, y)dx
\end{align*}

\sheader{Law of the Unconcious Statistician for 2 Random Variables} Let $X, Y$ be R.V.s and $g$ a function, then:
\begin{align*}
    \mathbb{E}g(x, y) = \iint g(x, y)f(x, y)dxdy && \mathbb{E}g(x, y) = \sum_x\sum_y g(x, y)f(x, y)
\end{align*}
This implies that $\mathbb{E}(X + Y) = \mathbb{E}X + \mathbb{E}Y$.
\gap
\sheader{Independence of Random Variables} Two independent random variables $X, Y$ are independent if
for any $a, b \in \mathbb{R}$,
\begin{align*}
    P\left(\curly{X \leq a} \cap \curly{Y \leq b}\right) = P\left(\curly{X \leq a}\right) P\left(\curly{Y \leq b}\right)
\end{align*}
It also follows that $X, Y$ are independent iff:
\begin{align*}
    p(x, y) &= p_X(x)p_Y(y)\\
    f(x, y) &= f_X(x)f_Y(y)
\end{align*}
Finally, one can also say that $X, Y$ are independent if the matrix of the values is full rank.
\gap
\sheader{Compositions of Functions with 2 Random Variables} Suppose that 
$X, Y$ are independent RVs. Then,
\begin{align*}
    \mathbb{E}\left(g(X)h(Y)\right) = \mathbb{E}g(X) \cdot \mathbb{E}h(Y).
\end{align*}
\sheader{Covariance} We can define the covariance of two RVs as: 
\begin{align*}
    \textrm{Cov}(X, Y) = \mathbb{E}(XY) - \mathbb{E}X \cdot \mathbb{E}Y.
\end{align*}
From this, it is implied that if $X, Y$ are independent, then $\textrm{Cov}(X, Y) = 0$,
but importantly, for only the forwards direction. Covariance describes the nature 
of random variables to grow or shrink together.
\gap
\sheader{Correlation Coefficient} The correlation coefficient is defined to be:
\begin{align*}
    \rho(X, Y) \equiv \frac{\textrm{Cov}(X, Y)}{\sqrt{\textrm{Var}(X) \textrm{Var}(Y)}}.
\end{align*}
It can be interpreted as a ``normalized'' covariance.
\gap
\sheader{Theorem (Cauchy Schwartz Inequality)}
\begin{align*}
    \left|\mathbb{E}(XY)\right|^2 \leq \mathbb{E}X^2 \cdot \mathbb{E}Y^2.
\end{align*}
\sheader{Sum of Variance} For two RVs $X, Y$, we can obtain the variance of their sum:
\begin{align*}
    \textrm{Var}(X + Y) = \textrm{Var}(X) + \textrm{Var}(Y) + 2\textrm{Cov}(X, Y),
\end{align*}
for which if $X, Y$ are independent, then the covariance term is zero, so
\begin{align*}
    \textrm{Var}(X + Y) = \textrm{Var}(X) + \textrm{Var}(Y)
\end{align*}
\sheader{Sums of Random Variables} For continuous RVs, we can find the
p.m.f. of their sum under the convolution:
\begin{align*}
    f_{X + Y}(a) = \int_{-\infty}^{\infty} f_X(a-y)f_Y(y)dy.
\end{align*}
\sheader{Gamma Distribution} We define another distribution, the gamma distribution, $X \sim \textrm{gamma}(n, \lambda)$ with 
p.m.f.:
\begin{align*}
    f(x) = \frac{\lambda^n x^{n - 1} e^{-\lambda x}}{(n - 1)!}
\end{align*}

\textbf{Midterm 1 Cutoff}

\pagebreak

\header{Poisson Processes}

Given some random variable $N_t$, which describes the number of 
events to occur by some time $t$, where each event happens sequentially,
and has the the time to complete of $X_i \sim \textrm{Exp}(\lambda)$,
this is called a Poisson Process:
\begin{itemize}
    \item $\mathbb{E}N_t = \lambda t$.
    \item $P(N_t = m) = \frac{(\lambda t)^m e^{-\lambda t}}{m!}$.
\end{itemize}
Alternatively, we can also say that $S_n$ is RV that represents the time 
it takes for the $n$-th event to occur. Hence, it follows that:
\begin{align*}
    P(S_N = T) = P(S_T = N).
\end{align*}

\header{Moment Generating Functions and Characteristic Functions}

We define the \underline{characteristic function} of a random variable 
to be the following:

\begin{align*}
    \phi(t) = \mathbb{E}e^{itx} = M(it) = \begin{cases}
        \sum_{n = - \infty}^\infty p(n)e^{itn}\\
        \int_{-\infty}^{\infty}e^{itx}f(x)dx.
    \end{cases}
\end{align*}

The terminology ``moment generating functions'' comes from the following 
property:
\begin{align*}
    \left[\frac{d^n}{dt^n}\right]_{t=0} \phi(t) = i^n \mathbb{E}X^n.
\end{align*}

We can also extract the pdf from a characteristic function:
\begin{align*}
    f(x) = \frac{1}{2\pi}\int_{-\infty}^{\infty} e^{-itx} \phi(t)dt.
\end{align*}

\sheader{Properties of Characteristic Functions}
\begin{itemize}
    \item $\ds\phi_{aX + b} (t) = e^{itb}\phi_X(at)$
    \item $P(X = c) = 1 \implies \phi_X(t) = e^{itc}$.
\end{itemize}

\sheader{Continuity Theorem} Let $X_1, X_2, \ldots $ be rvs with 
CDFs $F_1, F_2, \ldots$ and characteristic functions $\phi_1, \phi_2, \ldots$,
then:
\begin{enumerate}[(a)]
    \item If $F_n \to F$ where $F$ is the cdf of some rv $X_i$, 
    with char fn. $\phi$, then:
    \begin{align*}
        \lim_{n \to \infty}\phi_n(t) = \phi(t)\,\,\, \forall t \in \mathbb{R}.
    \end{align*}
    \item If $\ds \lim_{n \to \infty} \phi_n(t)= \phi(t)$ exists $\forall t \in \mathbb{R}$,
    then $\phi$ is the char. fn. of some RV $X$ (with cdf $F$), and 
    $F_n \to F$ and $X_n \to X$.
\end{enumerate}

\header{Large Things}

\sheader{(Weak) Law of Large Numbers} Let $X_i$ be i.i.d.. Assume 
$\mu = \mathbb{E}X_i < \infty$ and $Var(X_i) < \infty$. Let $S_n = X_1 + \cdots + X_n$.
Then:
\begin{align*}
    \frac{S_n}{n} \xrightarrow{\text{D}} \mu.
\end{align*}

\pagebreak

\sheader{Central Limit Theorem} Let $X_i$ be i.i.d. RVs with 
$\mathbb{E}X_i$ and $\textrm{Var}(X_i)$ finite. Let $S_n = X_1 + \cdots + X_n$.
Then,
\begin{align*}
    \frac{S_n - n \mu }{\sigma \sqrt{n}} \xrightarrow{D} N(0, 1)
\end{align*}

We can interpret this by saying that $S_n - n\mu $ has fluctuations of 
$\sigma \sqrt{n}$.
\gap
\sheader{Applying the Central Limit Theorem} Let $X_i$ be some random variable 
with $\mathbb{E}X_i = \mu$ and $\textrm{Var}(X_i) = \sigma^2$, and $S_n = \sum_{i=1}^{n}X_i$. 
Then, it follows that:
\begin{align*}
    P(S_n > a) \approx P\left(Z > \frac{a - n \mu}{\sigma \sqrt{n}}\right)
\end{align*}

\header{Statistical Estimation}

\sheader{Statistical Estimator} A statistical estimator is a 
function of the data:
\begin{itemize}
    \item \sheader{Sample Mean} $\ds \overline{X} = \frac{1}{n} \sum_{i = 1}^{n}X_i$
    \item \sheader{Sample Variance} $\ds S^2 = \frac{1}{n - 1} \sum_{min}^{max}(X_i - \overline{X})^2$.
\end{itemize}

\sheader{Unbiased Estimate} Given some data with parameter $\gamma$,
if an estimate from the data for $\gamma$ is $\hat{\gamma}$, then 
if $\mathbb{E}\hat{\gamma} = \gamma$, then $\hat{\gamma}$ is an 
unbiased estimate of $\gamma$.
\gap
\sheader{Hypothesis Testing} Given some hypothesis of a mean value $\mu = b$,
if $\mu$ is not in some confidence interval of probability $p$, then we 
reject the hypothesis that $\mu = b$.
\gap 
\sheader{Scenario 1} Given some i.i.d. data with known variance $\sigma^2$, but unknown mean $\mu$,
We can utilize the CLT to find the probability that the mean $\mu$ is in some
range, with a given probability $p$ (confidience interval).

\begin{align*}
    P(|Z| < a) = p \implies \overline{X} \in \left[\mu - \frac{\sigma}{\sqrt{n}} a, \mu + \frac{\sigma}{\sqrt{n}} a \right] \iff \mu \in \left[\overline{X} - \frac{\sigma}{\sqrt{n}} a, \overline{X} + \frac{\sigma}{\sqrt{n}}\right]
\end{align*}

\sheader{Scenario 2} Given some i.i.d. data with both unkonwn mean $\mu$ and 
variance $\sigma^2$, then we utilize the CLT and the students-t distribution 
to determine the significance of the data.

\begin{align*}
    P(|T| < a) = p \implies \mu \in \left[\overline{X} - a \frac{S}{\sqrt{n}}, \overline{X} + a \frac{S}{\sqrt{n}}\right].
\end{align*}

\header{Random Walks}

Given some random walk in where each step by $\pm \vec{e}_i$ in $\mathbb{R}^n$ is equally likley,
that is:
\begin{align*}
    P(\vec{x}_i = \vec{e}_j) = \frac{1}{2d}
\end{align*}

We say that the probability of returning to the origin is (where $M$ is the number of visits to the origin):
\begin{itemize}
    \item \sheader{Recurrent} if $u =1 \implies M = \infty$ (you always come back)
    \item \sheader{Transient} if $u < 1 \implies M < \infty$
\end{itemize}
It is known that a random walk in $\mathbb{Z}^d$ is recurrent for 
$d = \curly{1, 2}$ and transient for $d > 2$.

\pagebreak

\header{Gambler's Ruin}

Assume that a ``gambler'' has $k$ dollars and a ``banker'' has $b$ dollars.
They play games with probability $p$ that the gambler wins, and play 
until one of the two goes broke. It follows that for $k$ initial dollars
for the gambler, and $N = k + b$ total dollars, then:
\begin{align*}
    P(\textrm{gambler wins}) = \frac{\alpha^k - 1}{\alpha^N - 1} && \alpha = \frac{1-p}{p}
\end{align*}

\header{Conditional Probability}





\end{document}